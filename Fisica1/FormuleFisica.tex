\documentclass[a4paper,11pt]{article}

\title{Formulario di Fisica}
\author{}
\usepackage[utf8]{inputenc}
\usepackage{amsmath}
\usepackage{amssymb}
\usepackage{xfrac}
\usepackage[italian]{babel}
\usepackage{xifthen}
\usepackage{xparse}
\usepackage{color}
\usepackage{forloop}
\usepackage{braket}
\renewcommand*\rmdefault{ppl}
\newcommand{\qqquad}[0]{\qquad \qquad}
\newcommand{\norm}[1]{\mid #1 \mid}
\newcommand{\ang}[1]{\langle #1 \rangle}
\newcommand{\de}[0]{\mbox{d}}
\newcommand{\vers}[1]{\hat{#1}}

\begin{document}
\maketitle

\section*{Meccanica}
\subsection*{Dinamica}
\begin{itemize}
\item Accelerazione tangenziale date l'accelerazione e la velocit\`a: $\vec{a_T} = \vec{a} \cdot \vec{v}$
\item Accelerazione centripeta: $\vec{a_C} = \vec{a} - \vec{a_T}$
\item Raggio istantaneo di curvatura: $\rho = \frac{v^2}{a_C}$
\item Angolo tra due vettori: $\cos \theta_{AB} = \frac{\vec{A} \cdot \vec{B}}{\norm{\vec{A}}\norm{\vec{B}}}$
\end{itemize}

\subsection*{Attriti}
\begin{itemize}
\item Attrito Radente (Statico / Dinamico): $F_A = \mu N$
\item Attrito Volvente: $F_A = \mu_V \frac{N}{r}$
\item Attrito Viscoso su una sfera (Legge di Stokes): $F_V = 6\pi \eta r v$
\end{itemize}
Con $r$ raggio della sfera, $v$ velocit\`a dell'oggetto, $\mu$ coefficiente d'attrito, $\eta$ coefficiente di elasticit\`a, $N$ forza normale alla superficie.

\subsection*{Moto Armonico}
\subsubsection*{Semplice}
$F = -kx \implies x(t) = \frac{v_0}{\omega}\sin(\omega t) \quad \mbox{ con } \omega = \frac{k}{m}$

\subsubsection*{Smorzato}
Quando ho una forza $F = -kx-b\dot{x}$ cio\`e che si oppone alla velocit\`a. $2\gamma = \frac{b}{m}$, $\omega^2 = \frac{k}{m}$, $\Omega^2 = \gamma^2 - \omega^2$. La soluzione della equazione \`e: $x(t) = e^{-\gamma t} \left(Ae^{\Omega t} + Be^{-\Omega t} \right)$. \\
\\
{\it{Underdumping $\Omega^2 < 0$}} \\
$\bar{\omega} = \mathfrak{Im}(\Omega) \implies x(t) = e^{-\gamma t} C \sin(\bar\omega t + \varphi)$ \\
\\
{\it{Overdumping $\Omega^2 > 0$}} \\
$x(t) = Ae^{-(\gamma - \Omega)t}+Be^{-(\gamma + \Omega)t}$ \\
\\
{\it{Critical Dumping $\Omega^2 = 0$}} \\
$x(t) = e^{-\gamma t} (A + Bt)$ \\

\subsubsection*{Guidato}
Quando ho una forza $F = -kx-b\dot{x}+F_d\cos(\omega_d t)$ \\
Per l'equazione guardare il Morin (pagine 110-112).

\subsection*{Coordinate Polari}
\begin{itemize}
\item $\vec{r} = r \vers{r}$
\item $\dot\vec{r} = \dot{r}\vers{r} + r\dot{\theta}\vers{\theta}$
\item $\ddot\vec{r} = (\ddot{r} - r{\dot\theta}^2)\vers{r} + (2\dot{r}\dot\theta + r\ddot\theta)\vers\theta$
\end{itemize}

\subsection*{Macchine di Atwood}
\begin{enumerate}
\item Scrivi $F = ma$ per tutte le masse (con le loro tensioni).
\item Lega le accelerazioni delle varie masse notando che la lunghezza della corda non cambia.
\end{enumerate}

\subsection*{Problema di Keplero}
Nel seguito si intendono $b$ parametro d'impatto, $e$ eccentricit\`a della conica, $E$ energia del satellite, $L$ momento angolare del satellite, $a$ semiasse maggiore, $M$ massa dell'oggetto enorme, $m$ massa del satellite, $r$ distanza, $\varphi$ angolo, $T$ periodo dell'orbita. \\
$$ r = \frac{p}{1+e\cos\varphi} \qquad \mbox{ con } p = \frac{L^2}{GMm^2} \qquad e = \sqrt{1+\frac{2EL^2}{G^2M^2m^3}} = \sqrt{1 + \frac{p^2}{b^2}}$$
$$ E = - \frac{GMm}{2a} \qqquad \frac{T^2}{a^3} = \frac{4\pi^2}{GM} \mbox{ se } M >> m$$
Si ha la distanza minima ponendo $\varphi = 0$. Il satellite si allontana per $r \rightarrow +\infty$, ovvero $\cos\varphi = -\frac{1}{e}$. \\ Nel moto parabolico si arriva all'infinito con velocit\`a nulla.
Eccentricit\`a per vari tipi di coniche: Circonferenza $e=0$, Ellisse $0<e<1$, Parabola $e=1$, Iperbole $e>1$.

\subsection*{Casi in cui NON si conserva l'energia}
\begin{itemize}
\item Urti anelatici
\item Attriti
\item Forze esterne che compiono lavoro
\item Reazioni vincolari strane (spigoli retti e non arrotondati)
\end{itemize}

\subsection*{Leggi di conservazione standard}
\begin{itemize}
\item Energia
\item Momento angolare (soprattutto per gravit\`a)
\item Quantit\`a di moto (soprattutto per urti)
\end{itemize}

\subsection*{Formule ??}
\begin{itemize}
\item Energia potenziale elastica: $E = \frac{1}{2}k(\Delta x)^2$, con $\Delta x$ = spostamento dalla posizione di riposo.
\item Energia potenziale gravitazionale: $E = -G\frac{Mm}{r}$
\item Energia cinetica: $E = \frac{1}{2}mv^2$
\item Energia cinetica rotazionale: $E = \frac{1}{2}I\omega^2$
\item Momento angolare: $\vec{L_0} = \vec{r}\prod\vec{p}$
\item Momento torcente: $\vec{\tau_0} = \vec{r}\prod\vec{F}$
\item Quantit\`a di moto: $\vec{p} = m\vec{v}$
\end{itemize}

\subsection*{Centro di Massa}
\begin{itemize}
\item Tutto (Energia e Momento angolare) si spezza rispetto a quella del centro di massa pi\`u quella calcolata rispetto al centro di massa.
\end{itemize}

\section*{Termodinamica}
\subsection*{Formule a caso}
\begin{itemize}
\item Entropia per i Gas Perfetti: $S = nc_v \log T + nR \log \frac{V}{n}$
\item Relazione di Meyer: $C_P - C_V = R$
\item Legge dei gas perfetti: $pV = nRT$, $pV = NK_BT$, dove $N = nN_A = \mbox{ numero di molecole }$, $K_B$ costante di Boltzmann
\item Legge dei gas di Van der Waals: $(p+a\frac{n^2}{V^2})(\frac{V}{n}-b) = RT$
\item Calore assorbito (per i non-gas): $\mbox{d}Q = c m \mbox{d}T$, con $c$ calore specifico del corpo.
\item Entropia: $T \mbox{d}S = \mbox{d}U + p \mbox{d}V$, $\mbox{d}S = \left(\frac{\mbox{d}Q}{T}\right)_{\mbox{reversibile}}$
\item Calore assorbito (per i gas): $\mbox{d}Q = n c_v \mbox{d}T$
\item Primo principio della Termodinamica: $\mbox{d}U = \mbox{d}Q - \mbox{d}L$
\item Energia libera (o potenziale di Helmholz): $F = U - TS$
\item Entalpia: $H = U + PV$, $\Delta H < 0$ per trasformazioni spontanee.
\item Energia libera di Gibbs: $G = H - TS$
\end{itemize}

\subsection*{Altre formule a caso}
\begin{itemize}
\item Legge di Dalton: "In una miscela di gas la pressione totale \`e uguale alla somma delle pressioni parziali dei suoi gas componenti". $$P_{TOT} = \frac{RT}{V}\left( \sum_{i} n_i \right)$$
\item Forza media che {\bf una} molecola esercita sul contenitore cubico di lato $L$: $\norm{\vec{F}} = \frac{mv^2}{L}$
\item Forza totale esercitata dal gas: $\norm{\vec{F}} = \frac{1}{3}N \left(\frac{m \ang{v^2}}{L} \right)$, con $\ang{v^2}$ valor medio del quadrato della velocit\`a. $v_{qm} := (\ang{v^2})^{\frac{1}{2}}$ \`e la velocit\`a quadratica media.
\item $P = \frac{\norm{\vec{F_{TOT}}}}{L^2} = \frac{2}{3} N \left(\frac{1}{2}m {v_{qm}^{2}} \right) \frac{1}{V} = \frac{2}{3} N \ang{E_{cin}}$, $\ang{E_{cin}} = \frac{l}{2}K_BT = \frac{1}{2}m{v_{qm}^\frac{1}{2}}$, $l$ gradi di libert\`a.
\item Energia interna $U = \frac{l}{2}nRT$
\end{itemize}

\subsection*{Scambi di calore}
\begin{itemize}
\item Capacit\`a Termica di un corpo: $C = \frac{\Delta Q}{\Delta T}$
\item Conduzione: $\frac{Q}{\Delta t} = \frac{k A \Delta T}{d}$; $k$ conducibilit\`a termica, $A$ Area, $d$ spessore parete.
\item Irraggiamento: $\frac{\de E}{\de t} = \varepsilon \sigma A (\Delta T^4)$, $\varepsilon$ emissivit\`a, $\sigma$ costante di Stefan-Boltzmann.
\end{itemize}

\subsection*{Gradi di libert\`a}
\begin{itemize}
\item Gas Monoatomici: $l=3$.
\item Gas Biatomici: $l=5$.
\end{itemize}

\subsection*{Formule per le trasformazioni}
\begin{itemize}
\item Rendimento di un ciclo: $\eta = \frac{L}{Q_{ass}}$
\item Coefficiente di effetto frigogeno: $\mbox{COP} = \frac{Q_{\mbox{tolto al frigo}}}{L}$
\end{itemize}

\end{document}
