\documentclass[a4paper,NoNotes,GeneralMath]{stdmdoc}
\usepackage{tikz}
\usetikzlibrary{arrows,automata}

\def\checkmark{\tikz\fill[scale=0.4](0,.35) -- (.25,0) -- (1,.7) -- (.25,.15) -- cycle;} 
\newcommand{\crossmark}{$\times$}
\newcommand{\MCD}{\text{MCD }}
\newcommand{\Char}{\text{Char }}

\begin{document}
	\title{Teoria dei Campi e di Galois}

	\section*{Lemmi dimostrati}
	\begin{enumerate}
		\item ({\bf Formula del grado nelle torri di estensioni}) Siano $K \subseteq F \subseteq L$. Allora $[L:K] = [L:F] \cdot [F:K]$
		\item ({\bf Ogni estensione finita è algebrica}) Ma il viceversa è falso...
		\item ({\bf Fin.Gen. da elementi Algebrici $\implies$ Finita}) Se $F/K$ è finitamente generata da elementi algebrici, allora $F/K$ è finita
		\item ({\bf Esistenza del sottocampo fondamentale}) Ogni campo $K$ contiene un sottocampo fondamentale che può essere $\bbQ$, oppure $\bbF_p$
		\item ({\bf Teorema fondamentale dell'Algebra}) Il campo $\bbC$ è algebricamente chiuso
		\item ({\bf Esistenza ed unicità della chiusura algebrica}) Sia $K$ campo. Allora $\exists \bar{K}$ chiusura algebrica di $K$. Inoltre se $\bar{K}$ e $\bar{K'}$ sono chiusure algebriche di $K$ allora $\exists \phi: \bar{K} \rar \bar{K'}$ isomorfismo tale che $\phi \mid_K \equiv \Id$. Inoltre la se $K$ è infinito, la chiusura algebrica $\bar{K}$ ha la stessa cardinalità di $K$
		\item ({\bf Estensione di Omomorfismi}) $K \subseteq F$ e $\phi: K \rar L$ e $F/K$ algebrica. Allora $\exists \tilde{\phi}: F \rar L$ tale che $\tilde{\phi}\mid_K \equiv \phi$. Inoltre, se $F = K(\alpha)$ il numero di estensioni possibili di $\phi$ è uguale al numero di radici {\it distinte} di $\phi(\mu_\alpha)$ in $\bar{K}$.
		\item ({\bf Condizioni equivalenti alla normalità}) $F/K$ algebrica. Allora sono fatti equivalenti: \\
			\begin{itemize}
				\item $F/K$ è normale
				\item $\forall f \in K[x]$ irriducibile, se $f$ ha una radice in $F$ allora si spezza completamente in $F$
				\item $F$ è il campo di spezzamento di una famiglia di polinomi di $K[x]$
			\end{itemize}
		\item ({\bf Criterio della Derivata}) $f \in K[x], \Deg f \ge 1$. Allora vale: \\
			\begin{itemize}
				\item $f$ ha radici multiple $\sse$ $\MCD(f,f') \neq 1$
				\item $f$ irriducibile in $K[x]$. Allora $f$ ha radici multiple $\sse$ $f' \equiv 0$
			\end{itemize}
		\item ({\bf Corollario utile della Derivata}) $f \in K[x]$ irriducibile. Allora \\
			\begin{itemize}
				\item $\Char K = 0 \implies f$ è separabile
				\item $\Char K = p \implies $ Sia $r$ il massimo intero tale che $f(x) = g(x^{p^r})$ \\
					Allora si ha che ogni radice di $f$ ha molteplicità $p^r$, $g$ è irriducibile e separabile e gli zeri di $f$ sono le radici $p^r$-esime degli zeri di $g$
			\end{itemize}
		\item ({\bf Caratterizzazione dei campi perfetti}) Se $\Char K = 0$ oppure se $K = \bbF_{p^n}$ allora $K$ è perfetto. In realtà vale che, se $\Char K = p$, allora $K$ è perfetto se e solo se il morfismo di Frobenius ($x \mapsto x^p$) è surgettivo
		\item ({\bf Grado di separabilità su estensioni semplici}) $L = K(\alpha)$ con $\alpha$ algebrico su $K$ e di polinomio minimo $\mu_\alpha$. Allora \\
			\begin{itemize}
				\item $[K(\alpha):K]_S = \text{numero}\{\text{radici distinte di }\mu_\alpha\text{ in }\bar{K}\}$
				\item $\alpha$ è separabile su $K$ $\sse$ $[K(\alpha):K]_S = [K(\alpha):K]$
				\item Se $\Char K = p$ e $p^r$ è la molteplicità di $\alpha$ in $\mu_\alpha$ allora $[L:K] = p^r [L:K]_S$
			\end{itemize}
		\item ({\bf Il grado di separabilità è moltiplicativo}) $K \subseteq L \subseteq M$ algebriche. Allora $[M:K]_S = [M:L]_S \cdot [L:K]_S$
	\end{enumerate}

	\section*{Proprietà delle estensioni}
	\begin{tabular}{lcccc}
	{\bf Proprietà} & {\bf Torri} & {\bf Shift} & {\bf Composto} & {\bf Implica} \\
	Finita & \checkmark & \checkmark & \checkmark & Algebrica \\
	Algebrica & \checkmark & \checkmark & \checkmark & \\
	Normale & \crossmark & \checkmark & \checkmark & \\
	Separabile & & & & \\
	\end{tabular} \vskip 1cm

	\section*{Proprietà delle chiusure}
	\begin{tabular}{ll}
	{\bf Chiusura} & {\bf Proprietà varie} \\
	Chiusura Algebrica & Se $K$ è infinito $\bar{K}$ ha la stessa cardinalità \\
	Chiusura Normale & Se $F/K$ è finita anche $\tilde{F}/K$ è finita \\
	\end{tabular} \vskip 1cm
\end{document}
