\documentclass[a4paper,NoNotes,GeneralMath]{stdmdoc}
\usepackage[small,nohug,heads=vee]{diagrams}
\diagramstyle[labelstyle=\scriptstyle]
\usepackage{pgf}
\usepackage{tikz}
\usetikzlibrary{arrows,automata}
\newcommand{\Molt}{\text{molt }}
\newcommand{\Char}{\text{Char }}
\newcommand{\Sing}{\text{Sing }}
\newcommand{\hrar}{\hookrightarrow}
\newcommand{\xrar}{\xrightarrow}

\begin{document}
	\title{Fatti di EGA}
	
	\section*{Notazioni ed Introduzione}
	Il corso da cui sono tratti gli enunciati è diviso in alcune parti: nella prima si cerca di dare un'introduzione più concreta alla geometria algebrica attraverso anche esempi di curve in $\bbP^2$, nella seconda si parlerà di varietà quasi-proiettive, e di varietà affini e proiettive, nella terza ci sarà un po' di teoria della dimensione.

	\section*{Prima Parte}

	\subsection*{Studio dell'irriducibilità dei polinomi "quadratici"}	
	$p(x,y) = y^2 - f(x) \in \bbK[x][y]$. Se nella fattorizzazione di $f(x) = c \cdot p_1^{\alpha_1} \ldots p_k^{\alpha_k}$ con $p_i$ irriducibili e distinti, $\alpha_i > 0$ esiste un $i$ tale che $\alpha_i$ è dispari allora si ha $p(x,y)$ irriducibile. Inoltre se $\bbK$ è algebricamente chiuso questa condizione è anche necessaria.

	\subsection*{Studio locale delle Ipersuperfici Affini}
	$f \in \bbK[x_1, \ldots, x_n]$, $p \in V(f) \subseteq \bbA^n$. Sia $l$ retta di $\bbA^n$ passante per $p$, ovvero $l = \{p + tv \mid t \in \bbK \}$ con $v \in \bbK^n \setminus \{0\}$. \\

	Consideriamo il polinomio $g(t) := f(p + tv) \in \bbK[t]$ e distinguiamo due casi:
	\begin{itemize}
		\item $g \equiv 0$: Significa che la retta $l$ è contenuta in $V(f)$ e quindi diciamo che $l$ interseca $\cI_f$ in $p$ con molteplicità infinita.
		\item $g \not\equiv 0$, ma $g(0) = 0$ perché $p \in V(f)$. Quindi in $t=0$ ha una radice con una certa molteplicità $g(t) = t^m h(t)$ con $h(0) \neq 0$. Allora dico che $l$ interseca $\cI_f$ in $p$ con molteplicità $m$.
	\end{itemize}
	Se $m > 1$ diciamo che $l$ è tangente a $\cI_f$ in $p$. \\
	Invece diciamo che $p$ è un punto liscio o non singolare di $\cI_f$ se esiste almeno una retta $l$ che passa per $p$ e non è tangente. \\
	Fissato un punto $p$ vengono chiamate tangenti principali le rette tangenti che intersecano $\cI_f$ con molteplicità massima.

	\vskip 0.8cm
	
	In generale, a meno di una traslazione possiamo supporre $p = (0,0)$ e $p \in V(f)$. Allora considero una retta per l'origine $l= \{ tv \mid t \in \bbK \}$ e $g(t) := f(tv)$, con $v = (v_1, \ldots, v_n) \in \bbK^n \setminus \{0\}$. \\
	Allora $l$ è tangente a $f$ in $p$ $\sse g'(0) = 0$. $g'(t)\mid_{t=0} = \sum_{i=1}^n \dpar{f}{x_i} (tv) \cdot v_i \mid_{t=0} = \sum_{i=1}^{n} \dpar{f}{x_i} (p) \cdot v_i$ quindi $g'(0) = 0 \sse \sum_{i=1}^{n} \dpar{f}{x_i} (p) \cdot v_i = 0$ e distinguiamo dunque due casi:
	\begin{itemize}
		\item $\dpar{f}{x_i} (p) = 0 \quad \forall i$ allora $p$ è un punto singolare
		\item $\exists i \tc \dpar{f}{x_i} (p) \neq 0$ allora $p$ è liscio e l'insieme delle direazioni in $\bbK^n$ tangenti a $\cI_f$ in $p$ è un iperpiano di equazione $\sum_i \dpar{f}{x_i} (p) \cdot v_i = 0$
	\end{itemize}

	\vskip 0.8cm

	Inoltre, se scriviamo $f(x_1, \ldots, x_n) = f_m(\bm x) + h(\bm x)$ dove $f_m$ è omogeneo di grado $m \ge 1$ e tutti i monomi di $h$ hanno grado maggiore di $m$ allora abbiamo $\cI_f$ è liscia in $p$ $\sse$ $m = 1$ e inoltre sappiamo che ogni retta interseca $\cI_f$ in $p$ con molteplicità $\ge m$. E se il campo è infinito, per il principio di identità dei polinomi ho che $m$ è il minimo della molteplicità d'intersezione di $l$ con $\cI_f$ in $p$ al variare di $l$ tra le rette in $p$. Essa viene chiamata molteplicità del punto.
	Una retta si dice trasversale se $\Molt (l) = 1$.

	Si chiama cono tangente a $\cI_f$ in $p$ l'insieme delle rette che intersecano $\cI_f$ in $p$ con molteplicità maggiore del minimo $m$. è dato dall'equazione $f_m = 0$.

	Inoltre la molteplicità di $p$ per $\cI_f$ è uguale a $m$ $\sse$ tutte le derivate parziali di $f$ di ordine minore di $m$ si annullano in $p$ e c'è almeno una derivata parziale $m$-esima che non è nulla.

	Diciamo che un punto è un nodo se è singolare di molteplicità due.

	\subsection*{Omogenizzazione e Disomogeneizzazione}
	$D: \bbK[x_0, \ldots, x_n] \rar \bbK[x_1, \ldots, x_n]$ tale che $F(x_0, \ldots, x_n) \mapsto F(1, x_1, \ldots, x_n)$ che è ovviamente un omomorfismo di $\bbK$-algebre. \\
	$H: \bbK[x_1, \ldots, x_n] \rar \bbK[x_0, \ldots, x_n]$ che omogeneizza i polinomi, ovvero dato $f \neq 0$, $f \in \bbK[x_1, \ldots, x_n]$ sia $d = \Deg f$. Allora $H(f) := x_0^d \cdot f(\frac{x_1}{x_0}, \frac{x_2}{x_0}, \ldots, \frac{x_n}{x_0})$. Notiamo che $H$ NON è un omomorfismo però è moltiplicativo. \\
	Allora valgono:
	\begin{itemize}
		\item $H$ è moltiplicativo: $H(fg) = H(f) H(g)$
		\item $D \circ H = \Id$
		\item $H \circ D \mid_{\text{Polinomi Omogenei}} (F) = F_1$ con $F \in \bbK[x_0, \ldots, x_n]_d$ e vale $F = x_0^m F_1$ e $x_0 \nmid F_1$. Ovvero se $x_0 \mid F$ perdiamo le potenze di $x_0$ nel polinomio, altrimenti otteniamo la stessa cosa.
		\item $f \in \bbK[x_1, \ldots, x_n]$ irriducibile $\implies$ $F=H(f)$ irriducibile.
		\item $F \in \bbK[x_0, \ldots, x_n]$ irriducibile e $\neq x_0$ $\implies$ $f=D(F)$ irriducibile.
	\end{itemize}

	\subsection*{Fattorizzazione dei Polinomi Omogenei}
	Sia $F$ omogeneo, allora scrivo $F = x_0^m G$, con $G$ omogeneo e $x_0 \nmid G$. Considero allora $g := D(G) = D(F) \in \bbK[x_1, \ldots, x_n]$ e $g = c \cdot p_1^{\alpha_1} \ldots p_k^{\alpha_k}$ con i $p_i$ irriducibili distinti e $\alpha_i > 0$, $c \in \bbK^{*}$. Allora $P_i := H(p_i)$ che è ancora irriducibile e $F = x_0^m G = x_0^m H(g) = c x_0^m P_1^{\alpha_1} \ldots P_k^{\alpha_k}$. Quindi la fattorizzazione dei polinomi omogenei avviene in una variabile in meno ed i fattori di un polinomio omogeneo sono omogenei. \\
	Se ho $K$ algebricamente chiuso e $F(x_0, x_1)$ omogeneo di grado $d$, allora $F(x_0, x_1) = x_0^m G(x_0, x_1)$ con $G$ omogeneo e $x_0 \nmid G$. Allora $D(G) = g(x_1) = c \cdot \prod_{i=1}^{k} (a_i x_1 + b_i)^{\alpha_i}$ e allora $F(x_0, x_1) = c \cdot x_0^m \cdot H(g) = c \cdot x_0^m \cdot \prod_{i=1}^{k} (a_ix_1+b_ix_0)^{\alpha_i}$ e quindi se considero $[a_i, b_i] \in \bbP^1$ per $i=1, \ldots, k$ sono distinti e sono i punti in cui $F$ si annulla (oltre a $[0, 1]$ se $m > 0$) con molteplicità $\alpha_i$
	
	\subsection*{Punti singolari di $y^2 - p(x) = 0 \subseteq \bbA^2$}
	Sia $p$ un polinomio di $\Deg p = d \ge 3$ e $f(x,y) = y^2 - p(x)$. Troviamo i punti singolari del sottoinsieme di $\bbA^2$ dato da $f(x,y) = 0$. Serve necessariamente che (devono annullarsi tutte le derivate parziali) $\left\{ \begin{array}{c} y^2 = p(x) \\ y = 0 \\ p'(x) = 0 \end{array} \right.$ e quindi $\left\{ \begin{array}{c} y=0 \\ p(x) = 0 \\ p'(x) = 0 \end{array} \right.$ ovvero se e solo se $p$ ha radici multiple. Quindi i punti singolari sono quelli del tipo $(0,a)$ con $a$ radice multipla del polinomio $p$. \\
	Studiamo ora cosa avviene nei punti singolari: $f(x,y) = y^2 - (x-a)^\alpha q(x)$ con $\alpha \ge 2, q(\alpha) \neq 0$. Eseguiamo allora il cambio di coordinate affini $u := x - a, v := y$. $f(u,v) = v^2 - u^\alpha q_1(u)$ con $q_1(0) \neq 0$. La molteplicità allora è $2$. Inoltre se $\alpha = 2$ abbiamo un nodo, mentre se $\alpha > 2$, $v=0$ è l'unica tangente principale ed abbiamo quindi una cuspide. \\
	La chiusura proiettiva della curva è $F(x,y,z) = y^2z^{d-2} - P(x,z) = 0$. Vediamo i punti in cui $z = 0$ (cioè dove intersechiamo la retta all'infinito). $F(x,y,0) = - P(x, 0) = - a_d x^d = 0$ e quindi l'unico punto improprio è $x = 0, z = 0, y = 1$. Uso ora la carta affine $y \neq 0$ ed ottengo $z^{d-2} - P(x,z)$ e quindi se $d=3$ ho un punto liscio, se $d>3$ ho un punto singolare di molteplicità $d-2$ e l'unica tangente principale è $z=0$, se $d=3$ allora la molteplicità d'intersezione tra $z=0$ e il punto è $3$. $p = [0,1,0]$ è liscio e la retta tangente interseca $l$ in $p$ con molteplicità $3$, cioè $p$ è un flesso.
	
	\subsection*{Studio locale delle Ipersuperfici Proiettive}
	Lo facciamo passando alle carte affini: supponiamo di avere $[f]$ di $\bbA^n$ e ci associamo $[F]$ ipersuperficie proiettiva (detta chiusura proiettiva) $F = H(f)$ e inoltre data $[F]$ di $\bbP^n$ associamo $[D(F)]$ chiamato parte affine.
	
	\subsection*{Teorema di Eulero per le funzioni omogenee}
	$F \in K[x_0, \ldots, x_n]$ omogeneo di grado $d$. Allora vale che $d \cdot F (x) = \sum_{i=0}^{n} x_i \cdot \dpar{F}{x_i} (x)$
	
	\subsection*{Punti singolari di Ipersuperfici Proiettive}
	(Supponiamo $\Char K = 0$, anche se non sono sicuro che serva) Sia $p \in V(F) \subseteq \bbP^n$. $p = [1, a_1, \ldots, a_n] = [1, a]$. Sia $f = D(F) = F(1,x)$ allora $p$ è singolare per $F \sse \left\{ \begin{array}{c} f(a) = 0 \\ \dpar{f}{x_i}(a) = 0 \quad i = 1, \ldots, n \end{array} \right. \sse \left\{ \begin{array}{c} F(1,a) = 0 \\ \dpar{F}{x_i} (1,a) = 0 \quad i = 1, \ldots, n \end{array} \right.$ quindi mettere $x_0 = 1 $ prima o dopo aver derivato non fa nessuna differenza. Allora usando il teorema di Eulero si ha $\sse \dpar{F}{x_i}(p) = 0 \quad i=0,\ldots, n$
	
	\subsection*{Spazio tangente a $F$ in $a$ (applicato)}
	$\sum_{i=1}^{n} \dpar{f}{x_i} (a) \cdot (x_i - a_i) = 0$ è come fare $\sum_{i=1}^{n} \dpar{F}{x_i} (p) \cdot x_i - \sum_{i=1}^{n} \dpar{F}{x_i} (p) \cdot a_i$ e, supponendo che $p \in V(F)$ si ha (eulero) $ = \dpar{F}{x_0} (p) + \sum_{i=1}^n \dpar{F}{x_i}(p) \cdot x_i$ ovvero siccome la chiusura proiettiva si ottiene omogeneizzando con $x_0$ lo spazio tangente proiettivo è $\sum_{i=0}^n x_i \cdot \dpar{F}{x_i}(p) = 0$
	
	\subsection*{Teoria del Risultante}
	$A$ dominio d'integrità commutativo unitario. $F, G \in A[y]$, $F = a_0 + a_1 y + a_2 y^2 + \ldots + a_m y^m$, $G = b_0 + b_1 y + \ldots + b_n y^n$ dove $a_i, b_i \in A$ allora
	$$ Ris_y (F,G) := \Det \left( \begin{array}{ccccccccc}
	a_0		& a_1		& \ldots	& \ldots	& a_m		& 0	 		& 0			& \ldots	& 0		 \\
	0		& a_0		& \ldots	& \ldots	& \ldots	& a_m		& 0			& \ldots	& 0		 \\
	\vdots	& \vdots	& \vdots	& \vdots	& \vdots	& \vdots	& \vdots	& \vdots	& \vdots \\
	0		& \ldots	& 0			& a_0		& \ldots	& \ldots	& \ldots	& \ldots	& a_m	 \\
	\hline
	b_0		& b_1		& \ldots	& \ldots	& \ldots	& b_n		& 0			& 0			& 0		 \\
	0		& b_0		& \ldots	& \ldots	& \ldots	& \ldots	& b_n		& 0			& 0		 \\
	\vdots	& \vdots	& \vdots	& \vdots	& \vdots	& \vdots	& \vdots	& \vdots	& \vdots \\
	0		& \ldots	& 0			& b_0		& \ldots	& \ldots	& \ldots	& \ldots	& b_n	 \\
	\end{array} \right) $$

	\subsection*{Teorema di Bèzout}
	$\cC = [F], \cD = [G]$, con $m = \Deg \cC, n = \Deg \cD$, $K$ infinito. Allora si ha
	\begin{enumerate}
		\item Se il numero di intersezioni tra $\cC$ e $\cD$ è $> mn$ allora $\cC$ e $\cD$ hanno una componente in comune
		\item Se $K$ è algebricamente chiuso e $\cC$ e $\cD$ non hanno componenti in comune, allora $\cC \cap \cD$ consta di esattamente $mn$ punti se contati con molteplicità
	\end{enumerate}
	
	\subsection*{Corollari del Teorema di Bèzout}
	\begin{itemize}
		\item ($K$ algebricamente chiuso) $\cF \subseteq \bbP^n$ con $n \ge 2$ è un'ipersuperficie riducibile allora $\cF$ è singolare.
		\item ($K$ algebricamente chiuso) $\cC \subseteq \bbP^2$ una curva ridotta (ovvero nella fattorizzazione non compaiono componenti multiple) allora $\cC$ ha un numero finito di punti singolari.
		\item Siano $p_1, \ldots, p_5 \in \bbP^2$ cinque punti distinti. Quante coniche passano per $p_1, \ldots, p_5$?
		\item $p_1, \ldots, p_5 \in \cQ$ conica. Allora $p_1, \ldots, p_5$ sono in posizione generale $\sse$ $\cQ$ è liscia.
	\end{itemize}

	\subsection*{Definizione assiomatica di molteplicità d'intersezione tra due curve piane}
	$\cC = [f], \cD = [g] \subseteq \bbA^2, p \in \bbA^2$. Vorremmo definire la molteplicità dell'intersezione di $f$ e $g$ in $p$ $I(f \cap g, p)$ in modo che valgano:
	\begin{enumerate}
		\item $I(f \cap g, p) = +\infty \sse f,g$ hanno una componente in comune a cui $p$ appartiene
		\item $I(f \cap g, p) \in \bbN$ e $I(f \cap g, p) = 0 \sse p \not\in V(f) \cap V(g)$
		\item $I(f \cap g, p) = I(g \cap f, p)$
		\item $f, g$ rette distinte e $p \in V(f) \cap V(g)$ allora $I(g \cap f, p) = 1$
		\item $I(f \cap g, p)$ è invariante per affinità
		\item Dato $a \in K[x,y]$ si ha $I(f \cap g, p) = I(f \cap (g + af), p)$
		\item Se $f = \prod_i f_i$ e $g = \prod_j g_j$ allora deve valere che $I(f \cap g, p) = \sum_{i,j} I(f_i \cap g_j, p)$
	\end{enumerate}
	Queste proprietà determinano univocamente i numeri di intersezione. L'idea è, data una curva in $x$ e $y$ di abbassare il grado in $x$, supponendo che fino al grado $n-1$ i numeri di intersezione siano ben definiti e dimostrare che lo sono anche per $n$.
	
	\subsection*{Prima definizione di molteplicità d'intersezione}
	$p = (a,b)$ e si scompongano $f = f_1 a_1$, $g = g_1 b_1$ tali che $a_1 (p) \neq 0$, $b_1 (p) \neq 0$. Allora si ha $I(f \cap g, p) := \text{ molteplicità di } x = a \text{ come radice del risultante } \text{Ris }_y(f_1, g_1) \text{ in un sistema di coordinate generico }$
	
	\subsection*{Seconda definizione di molteplicità d'intersezione}
	$p = (a,b)$, $\cM_p = (x-a, y-b) \subseteq K[x,y]$. $\cM_p$ è il nucleo della $V_p: K[x,y] \rar K$ definita da $f \mapsto f(p)$ mappa di valutazione. $\cM_p$ è un ideale massimale. Allora localizziamo $\cO_p := K[x,y]_{\cM_p}$. Ora presi $f, g \in K[x,y]$ consideriamo la $K$-algebra $\frac{\cO_p}{(f,g)}$. Definiamo la molteplicità dell'intersezione come $I(f\cap g, p) = \Dim_K \frac{\cO_p}{(f,g)}$
	
	\subsection*{Quadriche di $\bbP^n$}
	Ci chiediamo quando siano singolari ($\Char K \neq 2$). Sia $x \in K^{n+1}$ e sia $Q(x) = {}^txAx = \sum A_{ij}x_ix_j$ con $A$ matrice $(n+1)\times (n+1)$ simmetrica e sia $p = [v] \in \bbP^n$. Allora notiamo che $\dpar{Q}{x_i}(v) = \sum a_{ij} v_j = (Av)_i$ e quindi $v$ è singolare per la quadrica $\sse \dpar{Q}{x_i}(v) = 0 \quad \forall i \sse Av = 0$. Quindi $\Sing Q = \bbP (\Ker A)$ la cui dimensione è $n - \Rk A$, ovvero $\cQ$ è liscia se e solo se ha rango massimo.
	
	\subsection*{Punti di flesso su curve proiettive}
	($\Char K \neq 2$) Sia $F$ curva di $\bbP^2$ e sia $f$ la sua parte affine. $(0,0) = p \in V(f)$. Vogliamo cercare una condizione affinchè $p$ sia un flesso. Supponiamo prima che $p$ sia un punto liscio. Scrivendo $f$ come "Somma di Taylor" si vede che i termini di grado $1$ e $2$ sono una conica affine e quindi vorremmo che la conica fosse riducibile per avere un punto di flesso. Quindi $p$ è di flesso $\sse$ il determinante dell'hessiano formale di $F$ è uguale a $0$. Siccome $\Deg \Det H(F) = 3d(d-2)$ i flessi sono abbastanza (per Bèzout). (E l'hessiano è identicamente nullo se e solo se $F$ è unione di rette)

	\subsection*{Cubica liscia in forma di Weierstrass}
	$\cC = [F]$ cubica liscia, $\Char K \neq 2,3$ e sia $O \in \cC$ flesso. Allora $\exists$ un sistema di coordinate omogenee $[z, x, y]$ su $\bbP^2$ tale che $O = [0, 0, 1]$ e $\cC$ ha equazione affine $y^2 = x^3 + ax + b$ con $\Delta = 4 a^3 + 27 b^2 \neq 0$ (Non stiamo supponendo $K$ algebricamente chiuso)
	
	\subsection*{Cubica liscia in forma di Legendre}
	Se $p(x) = x^3 + ax + b$ in forma di Weierstrass ha tutte le radici in $K$, allora $\cC$ può essere messa in forma di Legendre: $y^2 = x(x-1)(x-\lambda)$ con $\lambda \neq 0,1$
	
	\subsection*{Flessi di una cubica liscia su un campo algebricamente chiuso}
	$\cC$ cubica liscia e $K$ algebricamente chiuso. Scegliamo un flesso $O$ e mettiamo $\cC$ in forma di Weierstrass $y^2 = x^3 + ax + b = p(x)$ rispetto ad $O$. Cerco i punti di $\bbA^2$ in cui $\cC$ interseca $H(\cC)$: otteniamo $9$ flessi che sono tali che se $p_1, p_2 \in \cC$ sono flessi, allora la retta che passa per $p_1, p_2$ interseca $\cC$ in un terzo flesso. \\
	Inoltre il gruppo delle proiettività $g$ di $\bbP^2$ tali che $g \cC = \cC$ agiscono transitivamente sui punti di flesso. \\
	Abbiamo inoltre $12$ rette che passano per i punti di flesso e ogni retta passa per $3$ punti di flesso. I $9$ flessi e le $12$ rette che li congiungono formano una configurazione isomorfa al piano affine su $\bbF_3$.
	
	\subsection*{Birapporto, Proiettività e j-invariante}
	Ci chiediamo quando esiste una proiettività di $\bbP^1$ che porta una quaterna ordinata di punti in un'altra. Risposta: solo se hanno lo stesso birapporto. Siano $p_1, p_2, p_3, p_4 \in \bbP^1$ punti distinti e le $z_i = \frac{x_1}{x_0}$ le loro coordinate affini $\in K \cup \{ +\infty \}$. Dico che il birapporto è la coordinata affine di $z_4$ nel sistema di coordinate su $\bbP^1$ in cui $z_1 = 0, z_2 = +\infty, z_3 = 1$. Quindi $\text{Bir }(p_1, \ldots, p_4) = \frac{z_4 - z_1}{z_4 - z_2} \cdot \frac{z_3 - z_2}{z_3 - z_1}$ \\
	Vogliamo ora la condizione per quaterne non ordinate, quindi notiamo che permutando i punti si ottengono sei valori collegati del birapporto: $\{ \beta, \frac{1}{\beta}, 1-\beta , \frac{1}{1-\beta}, \frac{\beta}{1-\beta}, \frac{\beta - 1}{\beta} \}$ ovvero se e solo se hanno uguale $j$-invariante. $j: K \setminus \{0, 1\} \rar K$ definita da $j(t) = \frac{(t^2 - t + 1)^3}{t^2 (t-1)^2}$, dove il $j$-invariante viene calcolato sul birapporto delle quaterne. \\
	In realtà si può calcolare il birapporto anche sulle rette.
	
	\subsection*{Due cubiche liscie su un campo algebricamente chiuso sono proiettivamente equivalenti se e solo se hanno lo stesso $j$-invariante}
	
	\subsection*{Curve piane liscie su $\bbA^2_\bbC$}
	
	\subsection*{Sistema lineare di Curve}
	Fissato $d \ge 1$ il grado consideriamo $K[x_0, x_1, x_2]_d = \{ \text{polinomi omogenei di grado } d \} \cup \{ 0 \}$ che è uno spazio vettoriale su $K$ di dimensione $\binom{d+2}{2}$ e sia $V_d := \bbP ( K[x_0,x_1,x_2]_d )$, chiamato sistema lineare completo delle curve di grado $d$, che è uno spazio proiettivo i cui punti sono le curve piane di grado $d$. Un sistema lineare di curve di grado $d$ è un sottospazio proiettivo $W \subseteq V_d$. Se $\Dim W = 1$, $W$ si dice fascio.
	
	\subsection*{Imposizione del passaggio per un punto}
	$p = [a,b,c] \in \bbP^2$. $V_d (p) := \{ [F] \in V_d \mid F(p) = 0 \}$ è un iperpiano, sottospazio di $V_d$ definito da una equazione lineare. In generale posso fissare un po' di punti $p_1, \ldots, p_k \in \bbP^2$ ed ottenere $V_d(p_1, \ldots, p_k) := \cap_{i=1}^k V_d(p_i)$ che è un sistema lineare di dimensione che dipende da come sono disposti i punti ma ha codimensione al più $k$.
	
	\subsection*{Condizioni indipendenti per le Cubiche}
	($K$ infinito) Siano $p_1, \ldots, p_8 \in \bbP^2$ (anche coincidenti) tali che
	\begin{itemize}
		\item Non esiste una retta che contiene quattro dei $p_i$
		\item Non esiste una conica che passa per sette dei $p_i$
	\end{itemize}
	Allora $p_1, \ldots, p_8$ impongono condizioni indipendenti alle cubiche, cioè $\Dim V_3(p_1, \ldots, p_8) = 1$ \\
	Corollario: se ho due cubiche $\cC_1, \cC_2$ senza componenti comuni che si intersecano in $9$ punti distinti $p_1, \ldots, p_9$. Se $\cC$ è una cubica che passa per $p_1, \ldots, p_8$ allora $\cC$ passa anche per $p_9$.
	
	\section*{Seconda Parte: Varietà}
	\subsection*{Topologia di Zariski su $\bbA^n$}
	
	\subsection*{Topologia di Zariski su $\bbP^n$}
	
	\subsection*{Irriducibilità}
	\begin{itemize}
		\item $X \subseteq \bbA^n$ chiuso. Allora $X$ è irriducibile $\sse$ $I(X) \subseteq K[x_1, \ldots, x_n]$ è un ideale primo $\sse$ dati $U, V \subseteq X$ aperti non vuoti di $X$ si ha $U \cap V \neq \emptyset$
		\item $X$ irriducibile $\sse$ dati $U, V \subseteq X$ aperti non vuoti si ha che $U \cap V \neq \emptyset$. In particolare se $X$ è irriducibile ogni aperto è denso.
		\item $Y \subseteq X$. $Y$ irriducibile $\sse \bar{Y}$ irriducibile
		\item $Y \subseteq \bbP^n$ chiuso. Allora $Y$ è irriducibile $\sse \cC Y$ (il cono) è irriducibile in $\bbA^{n+1}$
	\end{itemize}
	
	\subsection*{Teorema di Fattorizzazione in irriducibili}
	Dato $Y \subseteq X$ chiuso una decomposizione in irriducibili di $Y$ è $Y = Z_1 \cup \ldots \cup Z_k$ con $Z_i$ chiusi irriducibili. La decomposizione si dice irridondante o minimale se $\forall i \neq j \quad Z_i \not\subseteq Z_j$. \\
	Negli spazi topologici Noetheriani $(X, \tau)$ ogni chiuso $Y \subseteq X$ ammette una decomposizione in irriducibili e, se minimale, essa è unica a meno di permutazioni degli irriducibili.
	
	\subsection*{Chiusi di $\bbA^1$ e di $\bbA^2$}
	I chiusi di $\bbA^1$ sono $\bbA^1$, $\emptyset$ e gli insiemi finiti di punti, ovvero La topologia di Zariski su $\bbA^1$ coincide con la cofinita. \\
	I chiusi di $\bbA^2$ sono unioni finite di punti e di ipersuperfici.
	
	\subsection*{Ipersuperfici}
	Con $K$ algebricamente chiuso intederemo ora per ipersuperficie il luogo di zeri di un'equazione e non più l'equazione stessa. Infatti se $X = V(f)$ ipersuperficie ($J = (f)$) allora se $f = p_1^{\alpha_1} \cdots p_k^{\alpha_k}$ con $p_i$ irriducibili e distinti, $\alpha_i \ge 0$, si ha $I(X) = \sqrt{(f)} = (p_1 \cdot \ldots \cdot p_k)$ e $V(f) = V(p_1) \cup \ldots \cup V(p_k)$ e, a meno di fattori multipli, il supporto le identifica univocamente.
	
	\subsection*{Chiusura proiettiva di chiusi algebrici}
	$X \subseteq \bbA^n$ chiuso. Allora la chiusura proiettiva è la chiusura di $X$ secondo Zariski nello spazio proiettivo $\bbP^n$ nel quale $\bbA^n$ è naturalmente immerso. Non basta omogeneizzare i generatori dell'ideale (vedi cubica gobba), serve prendere ogni elemento dell'ideale, omogeneizzaarlo e poi prendere l'ideale omogeneo generato. $I(\bar{X}) = (H(f), f \in I(X))$, ma se $X$ è un'ipersuperficie, allora ovviamente basta omogeneizzare la singola equazione (tanto le altre sono tutte sue multipli).
	
	\subsection*{Nullstellensatz}
	Se $K$ è un campo algebricamente chiuso, $J \subseteq K[x_1, \ldots, x_n]$ ideale. Allora le seguenti condizioni sono equivalenti:
	\begin{itemize}
		\item $V(J) = \emptyset \implies 1 \in J$
		\item $J$ massimale $\implies \exists p \in \bbA^n \tc I(p) = J$
		\item $I(V(J)) = \sqrt{J}$
	\end{itemize}
	Quindi nel caso di $K$ algebricamente chiuso abbiamo una corrispondenza biunivoca tra gli ideali radicali ed i chiusi di Zariski. Inoltre abbiamo anche le sottocorrispondenze $1:1$ tra ideali primi e chiusi irriducibili e tra ideali massimali e punti di $\bbA^n$
	
	\subsection*{Varietà Quasi-Proiettive}
	Seguono le varie definizioni:
	\begin{itemize}
		\item ({\bf Varietà Quasi-proiettiva}) É un localmente chiuso in uno spazio proiettivo, ovvero è intersezione di un chiuso e di un aperto. $Z \cap U \subseteq \bbP^n$ dove $Z$ è chiuso e $U$ è aperto.
		\item ({\bf Funzioni Regolari su VQP}) Data $X \subseteq \bbP^n$ VQP sia $f: X \rar K$. Allora $f$ si dice funzione regolare se $\forall p \in X \quad \exists U_p \subseteq X$ aperto tale che $\exists A,B \in K[x_0, \ldots, x_n] \tc A, B$ sono omogenei dello stesso grado con $B(q) \neq 0 \quad \forall q \in U_p$ e $f(q) = \frac{A(q)}{B(q)} \quad \forall q \in U_p$. (Notiamo che questo tipo di funzioni sono ben definite su $\bbP^n$, ovvero sono costanti sulle classi di equivalenza) \\
		La $K$-algebra delle funzioni regolari su $X$ si indica con $\cO_X(X)$
		\item ({\bf Morfismi di VQP}) Siano $X, Y$ due VQP e supponiamo di avere $f: X \rar Y$. Allora $f$ si dice morfismo se
		\begin{enumerate}
			\item $f$ è continua (Che è una richiesta piuttosto debole)
			\item $\forall V \subseteq Y$ aperto e $\phi: V \rar K$ regolare allora $\phi \circ f: f^{-1}(V) \rar K$ è regolare (che è una condizione di natura locale)
		\end{enumerate}
		Notiamo che l'identità è un morfismo e che i morfismi sono stabili per composizione. Diciamo che un morfismo di VQP è un isomorfismo se è biggettivo e la sua inversa insiemistica è anch'essa un morfismo di VQP
	\end{itemize}
	
	\subsection*{Varietà Affini}
	Sia $X \subseteq \bbA^n$ chiuso affine. Allora $X = \bar{X} \cap \bbA^n$ è una VQP attraverso l'identificazione di $\bbA^n$ con un sottoinsieme di $\bbP^n$. Notiamo che ora le funzioni regolari su $X$ diventano rapporti di polinomi non necessariamente omogenei, né dello stesso grado, ovvero $f: X \rar K$ allora $f$ è regolare se $\forall p \in X \quad \exists U_p \subseteq X$ intorno aperto e $a,b \in K[x_1, \ldots, x_n]$ tale che $b(q) \neq 0 \quad \forall q \in U_p$ e $f(q) = \frac{a(q)}{b(q)} \quad \forall q \in U_p$. \\
	Nel caso speciale in cui $b=1$ e $U = X$ $f$ viene detta funzione polinomiale. Attraverso $r_X: K[x_1, \ldots, x_n] \rar \cO_X(X)$ definita da $f \mapsto f\mid_X$ (che è un omomorfismo di $K$-algebre) notiamo che $\Ker r_X = I(X)$ e usando il primo teorema di isomorfismo abbiamo $K[X] := \frac{K[x_1, \ldots, x_n]}{I(X)} \hrar \cO_X(X)$ che viene detto anello delle coordinate di $X$ o algebra affine di $X$, molto importante per i chiusi affini su un campo algebricamente chiuso, poiché come vedremo caratterizza completamente i chiusi affini. \\
	Abbiamo una forma "Relativa" del Nullstellensatz, come corrispodenza $1:1$ tra gli ideali radicali di $K[X]$ e i sottoinsiemi chiusi $Y \subseteq X$.
	
	\subsection*{su $K$ algebricamente chiuso $r_X$ è un isomorfismo di $K$-algebre}
	$K$ algebricamente chiuso, allora $r_X : \frac{K[x_1, \ldots, x_n]}{I(X)} = K[X] \rar \cO_X(X)$ definita da $f \mapsto f\mid_X$ è un'isomorfismo di $K$-algebre. Ovviamente è un morfismo di $K$-algebre ed è iniettivo per come lo abbiamo costruito (quozientando sul ker). \\
	Sia allora $\phi \in \cO_X(X)$ e $J := \{ f \in K[X] \mid f\phi \in K[X] \} \subseteq K[X]$ ideale. Allora dico che $V(J) = \emptyset$ (da cui seguirebbe per NSS che $1 \in J$ e quindi $1 = \sum^{\text{Finita}} a_i(x) f_i(x)$ con $a_i \in K[X]$ e $f_i \in J$. Allora avrei $\phi = \sum_i a_i (f_i\phi)$ e sappiamo che $f_i \phi \in K[X]$ per definizione delle $f_i$). \\
	Fissiamo allora $p \in X$ e mostriamo che $p \not\in V(J)$. Decomponiamo per prima cosa $X$ in irriducibili e li separiamo in base a se contengono $p$ oppure no: $$ X = \underbrace{(X_1 \cup \ldots \cup X_k)}{\text{che contengono }p} \cup \underbrace{(X_{k+1} \cup \ldots \cup X_s)}{\text{che non contengono }p} $$
	Allora $\exists U_p$ aperto di $X$ tale che $p \in U_p$ e $a, b \in K[X]$ tali che $\phi\mid_{U_p} \equiv \frac{a}{b}$. Consideriamo la funzione $b\phi - a = 0$ su $U_p$ ma $U_p \cap X_i \neq \emptyset \quad \forall i = 1, \ldots, k$ e quindi $U_p$ è denso in $X_1 \cup \ldots \cup X_k$, da cui segue $b\phi - a = 0$ su $X_1 \cup \ldots \cup X_k$ perchè il luogo di zeri di una funzione regolare è un chiuso. \\
	Inoltre, siccome $p \not\in X_{k+1} \cup \ldots \cup X_s$ allora (per definizione di chiusi di Zariski) $\exists c \in K[X], c \in I(X_{k+1} \cup \ldots \cup X_s) \tc c(p) \neq 0$ allora $c(b\phi -a) \equiv 0$ su tutto $X$ ma allora $ca \in K[X]$ e si ha $ac = (cb)\phi$ e quindi $cb \in J$ ma allora $(cb)(p) \neq 0$
	
	\subsection*{Morfismi da una VQP in $\bbA^m$}
	Data $X \subseteq \bbP^n$ VQP vorrei descrivere i morfismi $X \xrar{f} \bbA^m$. Vale che $f: X \rar \bbA^m$ è un morfismo di VQP se e solo se le componenti di $f$ sono funzioni regolari.
	
	\subsection*{Varietà Affini}
	$X$ VQP si dice varietà affine se $X$ è isomorfo ad un chiuso di uno spazio affine. \\
	ATTENZIONE: Sia $X \subseteq \bbA^n$ chiuso e scegliamo $f \in K[X] \setminus 0$ e diciamo $X_f := \{ x \in X \mid f(x) \neq 0 \}$ è un aperto principale. Avevamo già osservato che gli aperti principali formano una base della topologia di Zariski di $X$. Mostriamo ora che $X_f$ è una varietà affine. (Basta "mandare gli zeri all'infinito", come nel Rabinowitsch trick) e quindi otteniamo il risultato che ogni VQP ha una base di aperti affini.
	
	\subsection*{Dualità Algebro-Geometrica}
	$f: X \rar Y$ morfismo di VQP. Allora $\exists f^{*}: \cO_Y(Y) \rar \cO_X(X)$ chiamato pullback definito da $\phi \mapsto \phi \circ f$ ed è un morfismo di $K$-algebre. \\
	Inoltre se $f$ è un isomorfismo di VQP allora $f^{*}$ è un isomorfismo di $K$-algebre.\\
	Notiamo che se $X$ è un chiuso affine allora $K[X]$ è una $K$-algebra finitamente generata e ridotta, ovvero senza nilpotenti. \\
	Ciò ci permette di determinare $K[X_f]$ per $X_f$ un aperto principale. (Usare l'isomorfismo dato dal fatto che gli aperti principali sono affini e passando alla star localizzare ad $f$)
	
	\subsection*{Funzioni Regolari su tutto $\bbP^n$}
	$K$ algebricamente chiuso, allora ogni funzione regolare su tutto $\bbP^n$ è costante. (Piuttosto agile, ad esempio su $\bbP^1$ considerare i due aperti principali con le loro equazioni $f = p(t) = q(\frac{1}{t})$)
	
	\subsection*{$K$ algebricamente chiuso, $\bbA^2 \setminus \{(0,0)\}$ non è una VQP}
	$K$ algebricamente chiuso. Se copriamo $X = \bbA^2 \setminus \{(0,0)\}$ con due aperti $U = \{x \neq 0\}$ e $V = \{y \neq 0\}$ che sono affini, si ha $K[U] = K[\bbA^2]_x = K[x, y]_x$ e $K[V] = K[x,y]_y$. Allora $\alpha: X \rar \bbA^2$ l'immersione mi da l'applicazione di pullback $\alpha^*$. Mostrando ora che è iniettiva e surgettiva vediamo che $X$ non è una VQP perché abbiamo un ideale fantasma ($M = (x,y)$ che è massimale, ma $V_X(M) = \emptyset$) e quindi fallisce il Nullstellensatz relativo.
		
	\subsection*{Lemmi e definizioni un po' casuali}
	\begin{itemize}
		\item Un morfismo $X \xrar{f} Y$ di VQP si dice dominante se la sua immagine è densa.
		\item Nel caso affine (se $X$ e $Y$ sono affini), $f$ è dominante se e solo se $f^{*}$ è iniettiva.
		\item In generale i morfismi non sono né aperti né chiusi.
		\item Diciamo che un insieme è costruibile se è un'unione finita di localmente chiusi. (Non lo dimostreremo, ma l'immagine di ogni morfismo è un costruibile)
		\item $\bbZ \subseteq \bbC$ è denso con la topologia di Zariski.
		\item $K[\bbA^n] \cong K[x_1, \ldots, x_n]$
		\item $X \subseteq \bbA^n, Y \subseteq \bbA^m$ chiusi. Allora $f: X \rar Y$ morfismo si dice immersione chiusa se $Z = f(X)$ è chiuso e $X \xrar{f} Z$ è un isomorfismo. \\
			Vale che $f$ è un'immersione chiusa $\sse f^{*}$ è surgettiva.
		\item $X, Y$ {\bf affini}. Allora $\phi: K[Y] \rar K[X]$ omomorfismo di $K$-algebre $\implies \exists ! f: X \rar Y$ morfismo tale che $\phi = f^*$
		\item $X, Y$ {\bf affini}, $f: X \rar Y$ morfismo. Allora $f$ è isomorfismo se e solo se $f^*$ lo è (insomma abbiamo il viceversa se le varietà sono affini)
	\end{itemize}
	
	\subsection*{Mappa di Veronese}
	$\cV_{1,2}: \bbP^1 \rar \bbP^2$ definita da $[x_0, x_1] \mapsto [x_0^2, x_0x_1, x_1^2]$ è ben definita, continua, ed è un morfismo che ha come immagine $\Img \cV_{1,2} = \{y_0y_2 - y_1^2 = 0\}$, viene detta mappa di Veronese. \\
	$\cV_{a,b}: \bbP^k \rar \bbP^N$ è la mappa di Veronese, dove $a$ è la dimensione dello spazio di partenza e $b$ è il grado dei monomi in arrivo e quindi $N = \binom{n+k}{k} - 1$ e su $\bbP^N$ abbiamo le coordinate $z_I$ dove $I$ è un multiindice di lunghezza $k$ e di grado $n$. La mappa di veronese è quindi definita da $\cV_{a,b} ([x_0, \ldots, x_k]) = [x^I]_I$ al variare di tutti i multiindici $I$ ed è un morfismo. \\
	Il fatto che gli $x^I$ commutino tra di loro ci dice quali sono le condizioni sulle coordinate immagine. Sia $\Sigma_{k,N} := \{ z_I z_J = z_{I'} z_{J'} \mid \forall I, J, I', J' \tc I+J = I'+J' \}$ che è quindi definito da una collezione di quadriche. È chiaro che $\Img \cV_{k,N} \subseteq \Sigma_{k,N}$ e definiamo l'inversa $g: \Sigma_{k,N} \rar \bbP^k$ mostrando che ci sono alcune coordinate che non si annullano mai e prendendo stringhe di queste... \\
	Tutta questa trafila serve per dimostrare che il complementare {\bf in $\bbP^n$} di un'ipersuperficie proiettiva è un chiuso affine, infatti la mappa di Veronese ci da un isomorfismo con l'immagine e il complementare di ipersuperfici nel proiettivo attraverso la mappa diventa un chiuso proiettivo tolto un piano (linearizza il polinomio) che è quindi un chiuso affine.
	
	\subsection*{VQP Prodotto}
	Vogliamo costruire il prodotto (in senso categorico) di due VQP $X$ e $Y$. Un prodotto di $X$ e $Y$ è una VQP $Z$ tale che $\exists p_1: Z \rar X, p_2: Z \rar Y$ morfismi tali che $\forall f: W \rar X, g: W \rar Y$ morfismi $\exists ! \phi: W \rar Z$ tale che il seguente diagramma commuti:
	\begin{diagram}
					&					&X					&					&\\
					&\ruTo^{f}			&					&\luTo^{p_1}		&\\
	W				&\rDashto^{\phi}	&					&					&Z\\
					&\rdTo^{g}			&					&\ldTo^{p_2}		&\\
					&					&Y					&					&
	\end{diagram}
	Basta dimostrare l'esistenza di $Z$ perché l'unicità è ovvia a meno di unico isomorfismo. Supponiamo infatti di avere due prodotti $(Z_1, p_1, p_2), (Z_2, q_1, q_2)$. Allora presi i due unici morfismi $\phi, \psi$ dati dall'essere prodotti si ha:
	\begin{diagram}
					&					&X					&					&\\
					&\ruTo^{p_1}		&					&\luTo^{p_1}		&\\
	Z_1				&\rDashto^{\psi\circ\phi}	&			&					&Z_1\\
					&\rdTo^{p_2}		&					&\ldTo^{p_2}		&\\
					&					&Y					&					&
	\end{diagram}
	E siccome anche l'identità fa commutare il diagramma si ha per unicità che $\psi\circ\phi = \Id$.
	\vskip 0.8cm
	Nel caso di varietà affini si ha $\bbA^n \times \bbA^m \cong \bbA^{m+n}$. Infatti prese le proiezioni naturali sulle componenti si verifica la proprietà universale sapendo che una mappa a valori in uno spazio affine è un morfismo se e solo se le sue componenti sono funzioni regolari. \\
	ATTENZIONE: $\bbA^{m+n}$ NON ha la topologia prodotto, ne ha una più fine (quella di Zariski su $\bbA^{m+n}$) \\
	Nel caso di varietà affini si verifica agilmente che se $X \subseteq \bbA^n, Y \subseteq \bbA^m$ sono due chiusi allora $X \times Y \subseteq \bbA^n \times \bbA^m$ è un chiuso ed è il prodotto di $X$ e di $Y$.\\
	L'anello delle coordinate del prodotto è $K[X \times Y] = K[X] \otimes_K K[Y]$
	Il prodotto di due spazi proiettivi si fa immergendoli in un proiettivo più grosso attraverso la mappa di Segre e si mostra che il morfismo così ottenuto è in realtà una biggezione con l'immagine. \\
	La mappa di Segre è: $S_{n,m}: \bbP^n \times \bbP^m \rar \bbP^{(n+1)(m+1)-1}$ definita da $([x_i],[y_j]) \mapsto [x_iy_j]$. e definito $\Sigma_{n,m} = \{ [z_{i,j}] \mid \Rk [z_{i,j}] \le 1 \}$ che si nota essere un chiuso in quanto intersezione di quadriche si hanno le due proiezioni / morfismi su $\bbP^n$ e su $\bbP^m$ dati da righe e colonne della matrice. \\
	Una base di chiusi del prodotto di due spazi proiettivi è data dal luogo di zeri di un polinomio biomogeneo (anche di gradi diversi). In particolare, anche in questo caso la topologia del prodotto è più fine della topologia prodotto.
	\vskip 0.8cm
	Il caso generale del prodotto di VQP segue in maniera semplice: siano $X, Y$ VQP. Allora $X = U\cap Z \subseteq \bbP^n$ e $Y = V \cap W \subseteq \bbP^m$ con $U,V$ aperti e $Z,W$ chiusi. Considero allora $X \times Y$ come sottoinsieme di $\bbP^n \times \bbP^m$ dato da $(U \times V) \cap (Z \times W)$ e si nota che $U \times V$ e $Z \times W$ sono rispettivamente aperto e chiuso in $\bbP^n \times \bbP^m$. Quindi $X \times Y$ è una VQP. Inoltre si vede anche che se $X, Y$ sono proiettive allora anche $X \times Y$ è una varietà proiettiva e che se $X,Y$ sono affini allora anche $X \times Y$ è affine.
	
	\subsection*{Quasi-T2 e proprietà del prodotto}
	Faremo un po' di striccheggi che in topologia generale si fanno se lo spazio è T2, ma qui ci riusciamo anche senza!
	\begin{itemize}
		\item ({\bf Diagonale chiusa nel prodotto}) $\Delta_X = \{ (x, x) \mid x \in X \} \subseteq X \times X$ è un chiuso
		\item ({\bf Due morfismi coincidono su un chiuso}) Siano $f,g: X \rar Y$ due morfismi. Allora $Z = \{ x \mid f(x) = g(x) \}$ è chiuso in $X$. (e quindi in particolare se due morfismi coincidono su un denso allora $f=g$)
	\end{itemize}
	Inoltre $X, Y$ irriducibili come VQP $\implies X \times Y$ è irriducibile (cosa che non è ovvia poiché la topologia del prodotto è molto fine)
	
	\subsection*{Cose casuali}
	\begin{enumerate}
	\item Diciamo che $G$ è un gruppo algebrico se
		\begin{itemize}
			\item G è una VQP
			\item G è un gruppo
			\item Le funzioni di inverso e di moltiplicazione sono morfismi
		\end{itemize}
	\item Se $X$ è una varietà proiettiva allora $P_2: X \times Y \rar Y$ (la proiezione) è una mappa chiusa $\forall Y$ VQP (Si dice che $X$ è universalmente chiusa)
	\item $X$ proiettiva. $f: X \rar Y$ morfismo. Allora $f$ è una mappa chiusa
	\item Se $X$ è proiettiva e connessa, $f: X \rar K$ è regolare allora $f$ è costante.
	\end{enumerate}	
	
	\section*{Terza Parte: Geometria Birazionale}
	\subsection*{Funzioni razionali (parziali)}
	
	
	\section*{Varie ed Eventuali}
	\subsection*{La Cubica Gobba}
	Fonte inesauribile di patologie e di controesempi. $\cC = \{ y-x^2 = z-xy = 0 \} \subseteq \bbA^3$ che è anche il grafico di $f: \bbA^1 \rar \bbA^2$ definita da $x \mapsto (x^2, x^3)$.
	
\end{document}
