\documentclass[a4paper,NoNotes,GeneralMath]{stdmdoc}
\usepackage{pgf}
\usepackage{tikz}
\usetikzlibrary{arrows,automata}
\newcommand{\MCD}{\text{MCD }}
\newcommand{\Lt}{\text{lt }}
\newcommand{\Lc}{\text{lc }}
\newcommand{\BdG}{\text{BdG }}
\newcommand{\Ris}{\text{Ris }}
\newcommand{\srar}{\twoheadrightarrow}
\newcommand{\hrar}{\hookrightarrow}
\newcommand{\Hom}{\text{Hom }}
\newcommand{\coKer}{\text{coKer }}
\newcommand{\gen}[1]{\langle {#1} \rangle}

\begin{document}
	\title{Algebra 2}
	
	\section*{Anelli}
	\begin{itemize}
		\item Se $A$ è un anello finito allora $A = A^* \sqcup \cD(A)$
		\item $f: A \rar B$ allora $\Img f \cong \frac{A}{\Ker f}$
		\item $I \subseteq A$ ideale, $B \subseteq A$ sottoanello allora vale $\frac{I+B}{I} \cong \frac{B}{I\cap B}$
		\item $I, J \subseteq A$ ideali e $I \subseteq J$. Allora vale $\frac{\frac{A}{I}}{\frac{J}{I}} \cong \frac{A}{J}$ \\
			Si ha inoltre la corrispondenza tra gli ideali di $\frac{A}{I}$ e gli ideali $J \subseteq A$ tali che $I \subseteq J$. In questa corrispondenza i primi ed i massimali si corrispondono
		\item $IJ \subseteq I \cap J$. Se vale $I + J = 1$ allora $IJ = I \cap J$
		\item È FALSO che $I \cap (J + K) = (I \cap J) + (I \cap K)$. FALSO
		\item $I \subseteq \sqrt{I}$
		\item ($A$ dominio) $a$ primo $\implies$ $a$ irriducibile
		\item ($A$ UFD) $a$ irriducibile $\implies$ $a$ primo
		\item Se $H \subseteq A \times B$ è ideale allora $H = I \times J$ con $I \subseteq A$, $J \subseteq B$ ideali
		\item $A \cong A_1 \times A_2 \sse \exists e \in A, e \neq 0,1 \quad e^2 = e$
		\item $\cD(A) = \cup_{a \notin A^*} (0 : a) = \cup_{a \notin A^*} \sqrt{(0 : a)}$ e $\sqrt{\cD(A)} = \cD(A)$, anche se non è necessariamente un ideale
		\item $\{ E_\lambda \}_{\lambda \in \Lambda}$ sottoinsiemi di $A$. Allora $\cup_{\lambda \in \Lambda} \sqrt{E_\lambda} = \sqrt{\cup_{\lambda \in \Lambda} E_\lambda}$
		\item Sia $A$ dominio con un numero infinito di elementi e $\mid A^* \mid < \infty$ allora $A$ possiede infiniti ideali massimali
		\item $I$ massimale $\implies I$ primo $\implies I$ primario. Inoltre $A$ dominio $\sse (0)$ ideale primo
		\item Sono equivalenti:
			\begin{itemize}
				\item $A$ ha un unico ideale massimale
				\item $\exists \mathfrak{m} \subseteq A$ ideale massimale $\tc \forall a \in A \setminus \mathfrak{m} \implies a \notin A^*$
				\item $\exists \km \subseteq A$ ideale massimale $\tc$ ogni elemento della forma $1 + \km$ è invertibile
			\end{itemize}
		\item $a \in \cJ(A) \sse \forall b \in A \quad 1-ab \in A^*$
		\item $\sqrt{I} = \cap_{I \subseteq P \text{ primi }} P$
		\item ({\bf Lemma di Scansamento}) $P_1, \ldots, P_n$ ideali primi. Sia $I \subseteq A$ ideale $\tc I \subseteq \cup_{i=1}^n P_i$. Allora $\exists j \tc I \subseteq P_j$
		\item $I_1, \ldots, I_n$ ideali e $P$ ideale primo. $\cap_{i=1}^n I_i \subseteq P \implies \exists j \tc I_j \subset P$. Inoltre se $P = \cap_i I_i$ allora $\exists j \tc I_j = P$
		\item ({\bf Teorema cinese}) Siano $I_1, \ldots, I_n \subseteq A$ ideali tali che $I_i + I_j = 1$. Allora $\forall a_1, \ldots, a_n \in A \quad \exists a \in A \tc a \equiv a_i (I_i)$
		\item $A$ anello c.u. Allora si ha che
			\begin{itemize}
				\item $f \in A[x]$ è un'unità $\sse$ $f = \sum_{i=0}^n a_i x^i$ con $a_i \in A$ tali che $a_0 \in A^*$ e $a_i \in \cN(A) \quad \forall i \ge 1$
				\item $f \in A[x]$ è nilpotente $\sse$ $\forall i \quad a_i \in \cN(A)$
				\item $f \in A[x]$ è divisore di zero $\sse$ $\exists c \in A, c \neq 0 \tc cf = 0$
			\end{itemize}
			Si ha inoltre per gli anelli di polinomi che
			\begin{itemize}
				\item $I$ primo $\sse I[x]$ primo
				\item $I$ primario $\sse I[x]$ primario
			\end{itemize}
			NON è vero che tutti gli ideali di $A[x]$ sono del tipo $I[x]$, come ad esempio $(x)$
		\item Gli ideali primi di $\bbZ[x]$ sono dei seguenti tipi:
			\begin{itemize}
				\item $(0)$
				\item $(p)[x]$ con $p \in \bbP$
				\item $(f(x))$ con $f$ irriducibile
				\item $(p, f(x))$ con $p \in \bbP$ e $f$ irriducibile modulo $p$ (Questi sono anche massimali)
			\end{itemize}
		\item $u \in A^*$, $a \in \cN(A)$, allora $u + a \in A^*$ (Somma di un nilpotente e di un invertibile è invertibile)
		\item $I$ primo $\implies I $ irriducibile
		\item In $A[x]$ si ha $\cN(A[x]) = \cJ(A[x])$ (Mentre in generale vale solo che $\cN(A) \subseteq \cJ(A)$)
		\item Sia $\phi: A \rar B$ omomorfismo di anelli. Allora
			\begin{itemize}
				\item $\phi(\cN(A)) \subseteq \cN(B)$
				\item Se $\phi$ è surgettivo allora $\phi(\cJ(A)) \subseteq \cJ(B)$
				\item $A$ semilocale (con un numero finito di ideali massimali) $\implies \phi(\cJ(A)) = \cJ(B)$
			\end{itemize}
		\item $A$ PID $\implies \cJ(A) = \cN(A)$
		\item $A \tc$ ogni ideale è primo $\implies$ $A$ è un campo
		\item $A \tc$ ogni ideale primo è principale $\implies A$ è un anello ad ideali principali
		\item $\sqrt{I}$ massimale $\implies I$ primario.
		\item $I$ primario, $J \not\subseteq \sqrt{I} \implies \sqrt{I : J^i} = \sqrt{I} \forall i$
		\item $I = \sqrt{I}$ e $h \notin I \implies I:h$ è radicale
		\item ({\bf Teorema della base di Hilbert}) Se $A$ è un anello Nötheriano, allora $A[x]$ è Nötheriano
	\end{itemize}
	
	\section*{Basi di Gröbner}
	\subsection*{Ideali Monomiali}
	Un ideale monomiale in $K[x_1, \ldots, x_n]$ è un ideale generato dai monomi
	\begin{itemize}
		\item ({\bf Criterio di appartenenza}) Sia $I$ un ideale monomiale e $f \in K[x_1, \ldots, x_n]$, $f = \sum_\beta c_\beta x^\beta$ con $c_\beta \in K$. Allora $f \in K \sse \forall \beta x^\beta \in I$
		\item ({\bf Lemma di Dickson}) Ogni ideale monomiale è finitamente generato. (La frontiera minimale di un ideale monomiale è unica, e viene detta Escalièr)
		\item ({\bf Operazioni con ideali monomiali}) Siano $I_1 = (m_1, \ldots, m_k)$ e $I_2 = (n_1, \ldots, n_s)$ con $m_i, n_j$ monomi. Allora si ha
			\begin{itemize}
				\item $I_1 + I_2 = (m_1, \ldots, m_k, n_1, \ldots, n_s)$
				\item $I_1 \cap I_2 = (\MCD_{i,j} (m_i, n_j))$
				\item $I_1 \cdot I_2 = (m_i \cdot n_j)_{i,j}$
				\item ({\bf Iatto}) $(I, m \cdot n) = (I, m) \cap (I, n)$ se $\MCD(m, n) = 1$ come monomi
				\item $I$ primo $\sse I = (x_{i_1}, \ldots, x_{i_k})$ (ed è massimale solo se le variabili compaiono tutte, ma DEVE essere monomiale)
				\item $I = \sqrt{I}$ (ovvero $I$ è radicale) $\sse \sqrt{m_i} = m_i \forall i$
				\item $I$ è primario $\sse I = (x_{i_1}^{\alpha_1}, \ldots, x_{i_k}^{\alpha_k}, m_1, \ldots, m_s)$ dove $m_1, \ldots, m_s \in K[x_{i_1}, \ldots, x_{i_k}]$
				\item $I$ è irriducibile $\sse$ $I = (x_{i_1}^{\alpha_1}, \ldots, x_{i_k}^{\alpha_k})$
				\item $I\cdot J = I \cap J \sse \forall i,j \quad \MCD(m_i, n_j) = 1$
				\item $I : J = \cap_i (I : n_i)$ e $I : (n_i) = (\frac{m_j}{\MCD(n_i, m_j)})_{j}$
			\end{itemize}
		\item Notare che usando la terza relazione del punto precedente possiamo spezzare ogni ideale monomiale in ideali primari e utilizzando $\sqrt{I \cap J} = \sqrt{I} \cap \sqrt{J}$ si possono calcolare anche gli ideali primi associati. \\
			Inoltre con la decomposizione in primari si calcolano bene i divisori di zero, i nilpotenti, etc.
	\end{itemize}
	
	\subsection*{Ordinamenti Monomiali Comuni}
	\begin{itemize}
		\item LEX $x_1 > x_2 > \ldots > x_n$. Dico che $\alpha \ge \beta \sse$ In $\alpha - \beta$ la prima coordinata $\neq 0$ è positiva
		\item DEGLEX Sia $\mid \alpha \mid := \sum_i \alpha_i$. Allora $\alpha \ge \beta \sse$ si ha $\mid \alpha \mid \ge \mid \beta \mid$ oppure $\mid \alpha \mid = \mid \beta \mid$ e vale $\alpha \ge \beta$ con LEX
		\item DEGREVLEX $\alpha \ge \beta \sse \mid \alpha \mid > \mid \beta \mid$ oppure si ha $\mid \alpha \mid = \mid \beta \mid$ e in $\alpha - \beta$ l'ultima coordinata $\neq 0$ è negativa
	\end{itemize}

	\subsection*{Basi di Gröbner e Algoritmo di Divisione}
	\begin{itemize}
		\item ({\bf Algoritmo di Divisione}) Siano $f_1, \ldots, f_k, f \in K[x_1, \ldots, x_n]$ allora $\exists a_1, \ldots, x_k, r \in K[x_1, \ldots, x_n]$ tali che $f = \sum_i a_i f_i + r$ e $\Deg (a_i f_i) \le \Deg(f)$. Inoltre se $r = \sum_\alpha r_\alpha x^\alpha$ si ha che se $r_\alpha \neq 0$ allora $x^\alpha \in (\Lt(f_1), \ldots, \Lt(f_k))$ \\
		Notiamo che posso fare dei passaggi "a mano" prima di partire con l'algoritmo di divisione e lui funzionerà comunque. La cosa importante è ricordarsi di soddisfare la condizione $\Deg (a_i f_i) \le \Deg(f)$ ad ogni passaggio.
		\item ({\bf Base di Gröbner}) Un insieme di polinomi $g_1, \ldots, g_k$ generatori di un ideale $I$ i cui leading term generano $\Lt(I)$ si dicono base di Gröbner. Sono equivalenti inoltre:
			\begin{itemize}
				\item $\forall f \quad \exists ! r$ resto della divisione di $f$ per $\{g_1, \ldots, g_k\}$
				\item $\forall f \in I = (g_1, \ldots, g_k)$ si ha $r = 0$ dall'algoritmo di divisione
				\item $\forall i,j \quad S(g_i, g_j)$ ha resto $r = 0$ nell'algoritmo di divisione
			\end{itemize}
		Dove per divisione si intende un risultato che soddisfi le ipotesi dell'algoritmo di divisione
		\item ({\bf Base di Gröbner ridotta}) Una BdG $G = \{ g_1, \ldots, g_k \}$ si dice ridotta se è minimale per inclusione e inoltre
			\begin{itemize}
				\item $\Lc(g_i) = 1 \quad \forall i$
				\item $(\Deg(g_1), \ldots, \Deg(g_k))$ sono un'escalièr per $\Deg(I)$
				\item $\forall g_i \quad g_i = \sum_\alpha c_\alpha x^\alpha$ allora $x^\alpha \notin \Lt(G \setminus \{g_i\})$
			\end{itemize}
		Teorema: La base ridotta è unica. Per ridurre una BdG basta prendere ciascun elemento $g$ ed effettuare la divisione per $G \setminus \{g\}$
		\item ({\bf S-polinomio}) Dati $f, g \in K[x_1, \ldots, x_n]$ e supponiamo $f = c_\alpha x^\alpha + f_1$ e $g = d_\beta x^\beta + g_1$ con $\Deg f = \alpha, \Deg g = \beta$. Allora dico S-polinomio tra $f, g$ il polinomio definito da $\gamma = (\gamma_1, \ldots, \gamma_n)$ con $\gamma_i = \max(\alpha_i, \beta_i)$
			$$ S(f, g) = \frac{x^\gamma}{c_\alpha x^\alpha} f - \frac{x^\gamma}{d_\beta x^\beta} g $$
	\end{itemize}
	
	\subsection*{Applicazioni e Computazioni}
	\begin{itemize}
		\item ({\bf Eliminazione di LEX}) $I \subseteq K[x_1, \ldots, x_n]$ allora $I_k = I \cap K[x_{k+1}, \ldots, x_n]$ è il $k$-esimo ideale di eliminazione. Vale il teorema: Se $G$ è una BdG rispetto a LEX con $x_1 \ge \ldots \ge x_n$ allora $\forall k = 1, \ldots, n-1$ si ha che $G_k = G \cap K[x_{k+1}, \ldots, x_n]$ è BdG di $I_k$
		\item ({\bf Cose calcolabili}) Dati $I, J \subseteq K[x_1, \ldots, x_n]$ e note le loro due BdG si ha
			\begin{itemize}
				\item ({\bf Intersezione}) $I \cap J = (tI, (1-t)J) \cap K[x_1, \ldots, x_n]$ dove quindi bisognerà usare l'ordinamento LEX con $t$ come variabile più pesante per poter usare eliminazione
				\item ({\bf Colon}) Se $\BdG(J) = \{h_1, \ldots, h_r\}$ allora $I : J = \cap_{i=1}^r (I : h_i)$. \\
					Se ora ho $f \in K[x_1, \ldots, x_n]$ e voglio calcolare $I : (f) = \{g \mid gf \in I\}$ allora ho che $I : (f) = \frac{1}{f} \cdot (I\cap (f))$, ovvero se $\BdG(I \cap (f)) = \{g_1 f, \ldots, g_k f\}$ allora ho $\BdG(I : (f)) = \{g_1, \ldots, g_k\}$
				\item ({\bf Ker di morfismi}) Sia $\Phi: K[x_1, \ldots, x_n] \rar K[y_1, \ldots, y_n]$ tale che $f_i(Y) := \Phi(x_i)$. Allora si ha $\Ker \Phi = (x_1 - f_1(Y), \ldots, x_n - f_n(Y)) \cap K[x_1, \ldots, x_n]$ ovvero bisogna calcolare l'ideale di eliminazione senza le $Y$
				\item ({\bf Appartenenza al radicale}) $f \in \sqrt{I} \sse 1 \in (I, 1-tf)$ e NON serve $K$ algebricamente chiuso
			\end{itemize}
		\item ({\bf Sistemi di equazioni polinomiali}) Cerchiamo le soluzioni comuni di $f_1 = 0, \ldots, f_n = 0$ in $K^n$. Valgono:
			\begin{itemize}
				\item ({\bf Esistenza di soluzioni}) Se $K$ è algebricamente chiuso, il sistema non ha soluzioni se e solo se $1 \in I = (f_1, \ldots, f_n)$, che si vede subito se c'è o meno con una BdG
				\item ({\bf Teorema di Estensione}) $I = (f_1, \ldots, f_k)$ e supponiamo $K$ algebricamente chiuso. $I_1 = I \cap K[x_2, \ldots, x_n]$ e $\beta \in \cV(I_1)$. $f_i = c_i(x_2, \ldots, x_n) \cdot x_1^{n_1} + \ldots \in K[x_2, \ldots, x_n][x_1]$. Se $\beta \notin \cV(c_1, \ldots, c_k)$ allora $\exists a \in K \tc (a, \beta) \in \cV(I)$. Ovvero se i termini davanti alle potenze più alte di $x_1$ non si annullano tutti su $\beta$ allora posso estendere $\beta$ ad una radice di $I$.
				\item ({\bf Conseguenza di Estensione}) Se la BdG è del tipo $\{x_1^{N_1} + \ldots, x_2^{N_2} + \ldots, \ldots, x_k^{N_k} + \ldots \}$ (deve essere di questa forma in tutte le variabili) allora la varietà è finita.
				\item ({\bf Soluzioni finite}) $K$ algebricamente chiuso. $I \subseteq A$. Allora sono fatti equivalenti:
					\begin{itemize}
						\item $\mid \cV(I) \mid < \infty$ ($\cV(I)$ è costituita da un numero finito di punti)
						\item $\forall i = 1, \ldots, n \quad \exists m_i \tc x_i^{m_i} \in \Lt(I)$
						\item $G = \{ g_1, \ldots, g_r\}$ BdG di $I$ allora $\forall i = 1, \ldots, n \quad \exists h_i \in \bbN \quad \exists g_r \in G \tc \Lt(g_r) \mid x_i^{h_i}$
						\item $\Dim_K \frac{A}{I} < \infty$
						\item $\Dim I = 0$ (come dimensione di Krull)
					\end{itemize}
					Inoltre vale che una $K$-base di $\frac{A}{I}$ è $\{x^\alpha \tc x^\alpha \notin \Lt(I)\}$, e anche $\Dim_K \frac{A}{I} = \mid \cV(I) \mid$ \\
					Osservazione: Il nullstellensatz serve solo per la freccia che $\mid \cV(I) \mid < \infty$ implica una delle altre. Per le freccie inverse non serve.
			\end{itemize}
	\end{itemize}
	
	\section*{Ideali e Varietà}
	Siano $I, J, H \subseteq K[x_1, \ldots, x_n]$ ideali e $V$ varietà affine. Allora vale
	\begin{itemize}
		\item $I \subseteq J \implies \cV(J) \subseteq \cV(I)$
		\item $I \subseteq \cI(\cV(I))$
		\item $\cV(\cI(V)) = V$
		\item $\cV(I) \subseteq \cV(J) \implies \cI(\cV(J)) \subseteq \cI(\cV(I))$
		\item $\cV(I + J) = \cV(I) \cap \cV(J)$
		\item $\cV(I\cdot J) = \cV(I) \cup \cV(J) = \cV(I \cap J)$
		\item $\cV(I) = \cV(\sqrt{I})$
		\item $\cV(I, JH) = \cV(I, J) \cup \cV(I, H)$
	\end{itemize}
	Valgono inoltre i seguenti fatti:
	\begin{itemize}
		\item $V$ è irriducibile $\implies \exists \kp \text{primo} \tc V = \cV(\kp)$ (il viceversa è vero se $K$ è algebricamente chiuso)
		\item Ogni varietà affine si decompone come unione di un numero finito di varietà irriducibili. Tale decomposizione si può minimizzare nel modo seguente: se compaiono due varietà irriducibili una contenuta dentro l'altra si toglie dall'unione la più piccola. La decomposizione minimalizzata è unica a meno dell'ordine con cui compaiono i fattori irriducibili
		\item $V = \{\alpha\}$ con $\alpha = (\alpha_1, \ldots, \alpha_n)$ allora $\cI(V) = (x_1 - \alpha_1, \ldots, x_n - \alpha_n)$ è un ideale massimale. (Se $K$ è algebricamente chiuso allora $I$ è massimale se e solo se è di quella forma)
		\item ({\bf Nullstellensatz}) $K$ algebricamente chiuso. Allora $I \subseteq K[x_1, \ldots, x_n]$ e si ha:
			\begin{itemize}
				\item ({\bf Forma debole}) $\cV(I) = \emptyset \sse 1 \in I$
				\item ({\bf Forma forte}) $\cI(\cV(I)) = \sqrt{I}$
			\end{itemize}
		\item ({\bf Normalizzazione di Nöther}) $K$ infinito. Se $f$ è un polinomio in $K[x_1, \ldots, x_n] \tc f \notin I_1 = K[x_2, \ldots, x_n]$ (ovvero $x_1$ compare) allora $\exists \phi$ cambio lineare di coordinate tale che $\phi(f) = c \cdot x_1^N + \overline{f}$ con $\Deg_{x_1} \overline{f} < N$ e $c \neq 0$ costante. 
		\item $K$ algebricamente chiuso. Se $I$ è radicale allora $I = \cap_{i=1}^k P_i$ con $P_i$ primi. (Basta decomporre la varietà)
	\end{itemize}
	
	\section*{Risultante}
	\begin{itemize}
		\item ({\bf Definizione di Risultante}) Sia $R$ un dominio d'integrità, $f, g \in R[x]$ e $f = \sum_{i=0}^n a_i x^i$, $g = \sum_{i=0}^m b_i x^i$. Definiamo allora la matrice di Sylvester come
		$$ \text{Sylv}(f, g) = \left[ \begin{array}{cccccccccc}
		a_0     & a_1     & \ldots  & \ldots  & a_n     & 0       & \ldots  & \ldots  & \ldots  & 0       \\
		0       & a_0     & a_1     & \ldots  & \ldots  & a_n     & 0       & \ldots  & \ldots  & 0       \\
		\vdots  &         & \ddots  &         &         &         & \ddots  &         &         & \vdots  \\
		0       & \ldots  & 0       & a_0     & a_1     & \ldots  & \ldots  & a_n     & \ldots  & 0       \\ \hline
		b_0     & b_1     & \ldots  & b_m     & 0       & \ldots  & \ldots  & \ldots  & \ldots  & 0       \\
		0       & b_0     & b_1     & \ldots  & b_m     & 0       & \ldots  & \ldots  & \ldots  & 0       \\
		0       & 0       & b_0     & b_1     & \ldots  & b_m     & 0       & \ldots  & \ldots  & 0       \\
		\vdots  &         &         & \ddots  &         &         &         & \ddots  &         & \vdots  \\
		0       & \ldots  & \ldots  & 0       & b_0     & b_1     & \ldots  & \ldots  & b_m     & 0       \\
		\end{array} \right]$$
		Ed il risultante di $f$ e $g$ è $\Ris(f, g) = \Det \text{Sylv}(f, g)$
		\item ({\bf Definizione alternativa}) $\Ris(f, g) = a_n^m b_m^n \prod_{i,j} (\alpha_i - \beta_j) = a_n^m \cdot \prod_{f(\alpha_i) = 0} g(\alpha_i) = (-1)^{mn} b_m^n \cdot \prod_{g(\beta_j) = 0} f(\beta_j)$ dove le $\alpha_i$ e le $\beta_j$ sono le radici rispettivamente di $f$ e di $g$, con molteplicità
		\item ({\bf Proprietà del risultante}) Valgono le seguenti proprietà:
			\begin{itemize}
				\item $\Ris(f, g) = (-1)^{mn} \Ris(g, f)$
				\item $\Ris(af, g) = a^n \Ris(f, g)$ con $a \in R$ scalare
				\item $\Ris(f, ag) = a^m \Ris(f, g)$ con $a \in R$ scalare
				\item $\Ris(a, b) = 1$ dove $a, b \in R$ sono scalari
				\item $\Ris(f, g) = 0 \sse \exists \alpha \in \overline{R} \tc f(\alpha) = g(\alpha) = 0$ (ovvero il risultante è nullo se e solo se $f$ e $g$ hanno una radice in comune nella chiusura algebrica del campo delle frazioni di $R$). Inoltre, se $R$ è UFD allora le due precedenti sono equivalenti a $\exists h \in R[x] \tc \Deg h > 0, h \mid f, h \mid g$
				\item $f, g \in R[x]$ e $\Deg f = n, \Deg g = m$, allora $\Ris(f,g) = Af + Bg$ con $A, B \in R[x]$ e $\Deg A < m, \Deg B < n$
				\item $\Ris(f, h_1 \cdot h_2) = \Ris(f, h_1) \cdot \Ris(f, h_2)$
				\item $\Ris(f, hf+g) = a_m^{\Deg (hf + g) \cdot \Deg g} \cdot \Ris(f, g)$ [ATTENZIONE: della formula a fianco non sono completamente sicuro]
				\item In molti casi vale che $\Ris(f,g) \mid_\alpha = \Ris(f\mid_\alpha, g\mid_\alpha)$ dove con $\mid_\alpha$ si intende la valutazione in $\alpha$. Bisogna solo stare attenti che almeno uno dei coefficienti direttivi valutati sia non nullo, altrimenti cambia la dimensione della matrice di sylvester e di conseguenza anche il polinomio che definisce il risultante
				\item Può essere comodo sapere che, detti $a_i$ e $b_j$ i coefficienti di $f$ e di $g$, si ha che $\Ris(f, g) \in \bbZ[a_i, b_j]$
			\end{itemize}
		\item ({\bf Trucchi utili con il risultante}) Dati $f = \prod_i (x - \alpha_i)$ e $g = \prod_j (x - \beta_j)$, allora si possono costruire i seguenti polinomi:
			\begin{itemize}
				\item $\Ris_y (f(x-y), g(y))$ ha radici $\gamma_{i,j} = \alpha_i + \beta_j$
				\item $\Ris_y (f(x+y), g(y))$ ha radici $\gamma_{i,j} = \alpha_i - \beta_j$
				\item $\Ris_y (y^{\Deg f} f(\frac{x}{y}), g(y))$ ha radici $\gamma_{i,j} = \alpha_i \cdot \beta_j$
				\item Se $g(0) \neq 0$ allora $\Ris_y (f(xy), g(y))$ ha radici $\gamma_{i,j} = \frac{\alpha_i}{\beta_j}$
			\end{itemize}
	\end{itemize}
	
	\section*{Moduli}
	\subsection*{Primi fatti}
	\begin{itemize}
		\item ({\bf Fregatura dei Moduli}) Attenzione che le seguenti cose non sono sempre vere su moduli generici:
			\begin{itemize}
				\item Non sempre esiste una base
				\item Un sistema di generatori minimale non è necessariamente una base
				\item Un insieme libero massimale non è necessariamente una base
				\item Due sistemi di generatori minimali non hanno necessariamente la stessa cardinalità (e nemmeno gli insiemi liberi massimali)
			\end{itemize}
		\item ({\bf Omomorfismi di $A$-Moduli}) Dati due $A$-Moduli $M$ ed $N$, allora si ha che anche $\Hom_A(M, N)$ è un $A$-modulo con le operazioni di somma e di prodotto scalare effettuate in arrivo. (Notare che questa proprietà è particolarmente strana e ci tornerà utile più volte). \\
		Inoltre si può notare come dato un omomorfismo $f: M \rar N$ di $A$-moduli si ha che $\Ker f = \{ m \in M \mid f(m) = 0 \}$ ed $\Img f = \{ f(m) \mid m \in M \}$ sono entrambi due sottomoduli rispettivamente di $M$ e di $N$. Allora possiamo anche sempre definire $\coKer f = \frac{N}{\Img f}$
		\item ({\bf Fatti di base e definizioni di operazioni importanti}) Valgono le seguenti cose:
			\begin{itemize}
				\item $\Hom_A(A, M) \cong_{\text{A-mod}} M$. Infatti conoscere il valore di $f(1)$ caratterizza tutto l'omomorfismo $f$, visto che è di $A$-moduli
				\item $L \subseteq N \subseteq M$ allora vale $\frac{M}{N} \cong_{\text{A-mod}} \frac{\frac{M}{L}}{\frac{N}{L}}$
				\item $M_1, M_2 \subseteq M$ sottomoduli. $M_1 + M_2 := \{ m_1 + m_2 \mid m_1 \in M_1, m_2 \in M_2 \}$ allora vale che $\frac{M_1 + M_2}{M_2} \cong_{\text{A-mod}} \frac{M_1}{M_1 \cap M_2}$
				\item ({\bf $\frac{A}{I}$-moduli}) Dato $I \subseteq A$ idale ed $M$ modulo si può definire $IM = \{ \sum_i a_i m_i \mid a_i \in I, m_i \in M \}$ e si verifica che è un sottomodulo di $M$. Inoltre vale che $\frac{M}{IM}$ è anche un $\frac{A}{I}$-modulo. \\
				Possiamo invece notare che $M$ non è sempre un $\frac{A}{I}$-modulo. Ci possiamo però riuscire se $I \subseteq (0 : M) = \{ a \in A \mid aM \subseteq (0) \}$.
				\item ({\bf Somma diretta e prodotto}) Dati $\{M_i\}_{i \in I}$ una famiglia di $A$-moduli si definisce $$\oplus_i M_i = \{ (a_i)_{i \in I} \mid a_i \in M_i, a_i \neq 0 \text{ solo per un numero finito di indici}\}$$ Inoltre si definisce $$\prod_i M_i = \{ (a_i)_{i \in I} \mid a_i \in M_i \}$$ senza la condizione di sopra. \\
				Se l'insieme $I$ di indici è finito allora si ha che $\oplus_i M_i = \prod_i M_i$. Valgono inoltre le seguenti proprietà universali per somma diretta e prodotto:
				\begin{itemize}
					\item Dati $\{M_i\}_{i \in I}$ $A$-moduli, si hanno $M_i \hrar^{j_i} \oplus_i M_i$ date da $m_i \mapsto (0, \ldots, 0, m_i, 0, \ldots)$. Allora per ogni assegnamento di $\{\varphi_i\}_{i \in I}$ con $\varphi_i: M_i \rar N$ omomorfismi di $A$-moduli, esiste unico $\tilde\phi: \oplus_i M_i \rar N$ tale che $\varphi_i = \tilde\phi \circ j_i$
					\item Dati $\{M_i\}_{i \in I}$ $A$-moduli, si hanno $\prod_i M_i \srar^{\pi_i} M_i$ le proiezioni date da $m = (m_j)_{j \in I} \mapsto m_i$. Allora per ogni assegnamento di $\{\varphi_i\}_{i \in I}$ con $\varphi_i: N \rar M_i$ omomorfismi di $A$-moduli, esiste unico $\tilde\phi: N \rar \prod_i M_i$ tale che $\varphi_i = \pi_i \circ \tilde\phi$
				\end{itemize}
			\end{itemize}
		\item ({\bf Morfismi da un modulo libero}) Sia $M$ un $A$-modulo libero e sia $S = \{s_1, \ldots, s_k\}$ una sua base. Allora dati $n_1, \ldots, n_k \in N$ ($N$ è un altro $A$-modulo) si ha che $\exists ! \Phi: M \rar N$ tale che $\Phi(s_i) = n_i$, $\Phi$ morfismo di $A$-moduli
		\item ({\bf Rango di un modulo libero}) Sia $M$ un $A$-modulo libero con base $B = \{b_1, \ldots, b_k\}$ finita. Allora ogni altra base di $M$ ha cardinalità $k$. Se $M$ è libero con base di cardinalità $k$ si dice che $M$ ha rango $k$ ($\Rk M = k$)
		\item $\Hom_A(A^n, M) \cong M^n$.
		\item $M$ è un $A$-modulo finitamente generato $\sse M \cong \frac{A^k}{\Ker \varphi}$ per un certo $k \in \bbN$ e per un certo $\varphi$. Se $M = \gen{m_1, \ldots, m_k}$ si ha $\varphi: A^k \rar M$ definito da $e_i \mapsto m_i$. Allora $M \cong \frac{A^k}{\Ker \varphi}$. Il viceversa è ovvio.
		\item ({\bf Hamilton-Cayley}) Sia $M$ un $A$-modulo finitamente generato, $I \subseteq A$ ideale. Sia $\varphi \in \Hom_A(M, M)$ endomorfismo tale che $\phi(M) \subseteq IM$. Allora $\exists b_0, \ldots, b_{n-1} \in I \tc \phi^n + \sum_{i=0}^{n-1} a_i \phi^i = 0$ in $\Hom_A(M, M)$
		\item ({\bf Nakayama}) Come corollario di Hamilton-Cayley si ottengono le seguenti tre versioni di Nakayama:
			\begin{itemize}
				\item Sia $M$ un $A$-modulo finitamente generato, $I \subseteq A$ ideale $\tc M = IM$. Allora $\exists a \in A \tc a \equiv 1 (\mod I)$ e $a \cdot M = 0$ (Basta applicare HC a $\varphi = \Id$)
				\item Sia $M$ un $A$-modulo finitamente generato, $\cJ(A)$ radicale di Jacobson, $I \subseteq \cJ(A)$ ideale di $A$ tale che $M = IM$. Allora $M = 0$ (Usiamo il Nakayama precedente ed usiamo la caratterizzazione del radicale di Jacobson)
				\item Sia $M$ un $A$-modulo finitamente generato, $N$ un sottomodulo, $I \subseteq \cJ(A)$ ideale di $A$. Se $M = N + IM$ allora $M = N$ (Usando il Nakayama precedente basta mostrare che $\frac{M}{N} = I(\frac{M}{N})$ così che $\frac{M}{N} = (0) \implies M = N$ e questo è piuttosto semplice)
			\end{itemize}
			Come corollario otteniamo che se $A$ è un anello locale e $\km$ un suo ideale massimale, $M$ un $A$-modulo finitamente generato. Allora se $n_1, \ldots, n_k$ sono elementi di $M$ tali che si ha che $\overline{n_1}, \ldots, \overline{n_k}$ generato $\frac{M}{\km M}$ allora $n_1, \ldots, n_k$ generano $M$ (considerare $N \hrar M \srar \frac{M}{\km M}$
		\item Sia $M$ un $A$-modulo finitamente generato, $f \in \End_A(M)$ surgettivo $\implies f$ è un isomorfismo.
		\item ({\bf Funtori $f^*$ e $g_*$}) Se ho $f: P \rar M$ allora posso considerare $f^*: \Hom_A(M, N) \rar \Hom_A(P, N)$ definito da $\phi \mapsto \phi \circ f$. Notiamo che è contravariante. \\
		Inoltre dato $g: M \rar P$ si ha $g_*: \Hom_A(N, M) \rar \Hom_A(N, P)$ definito da $\psi \mapsto g \circ \psi$, che è covariante. 
	\end{itemize}
	
	\subsection*{Omomorfismi tra moduli liberi e forma normale di Schmidt}
	\begin{itemize}
		\item Ogni elemento di $\Hom_A(A^m, A^n)$ si può rappresentare in modo unico come matrice, quindi mi basta sapere dove vanno gli $e_i$ base di $A^m$ per sapere dove vanno tutti gli altri elementi. Inoltre una matrice sarà invertibile se e solo se il suo determinante è un elemento invertibile dell'anello (Basta usare l'aggiunta sapendo che $M M^* = (\Det M) \Id$)
	\end{itemize}
	
	\subsection*{Successioni Esatte di Moduli}
	\begin{itemize}
		\item La successione $M_1 \rar^f M \rar^g M_2 \rar 0$ è esatta $\sse$ la successione $0 \rar \Hom_A(M_2, N) \rar^{g^*} \Hom_A(M, N) \rar^{f^*} \Hom_A(M_1, N)$ è esatta $\forall N$ $A$-moduli.
		\item La successione $0 \rar M_1 \rar^f M \rar^g M_2$ è esatta $\sse$ la successione $\Hom_A(N, M_1) \rar^{f_*} \Hom_A(N, M) \rar^{g^*} \Hom_A(N, M_2) \rar 0$ è esatta $\forall N$ $A$-moduli.
	\end{itemize}
\end{document}
