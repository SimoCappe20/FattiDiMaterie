\documentclass[a4paper,NoNotes,GeneralMath]{stdmdoc}
\usepackage{pgf}
\usepackage{tikz}
\usetikzlibrary{arrows,automata}
\newcommand{\MCD}{\text{MCD }}
\newcommand{\Lt}{\text{lt }}
\newcommand{\Lc}{\text{lc }}
\newcommand{\BdG}{\text{BdG }}

\begin{document}
	\title{Algebra 2}
	
	\section*{Anelli}
	\begin{itemize}
		\item Se $A$ è un anello finito allora $A = A^* \sqcup \cD(A)$
		\item $f: A \rar B$ allora $\Img f \cong \frac{A}{\Ker f}$
		\item $I \subseteq A$ ideale, $B \subseteq A$ sottoanello allora vale $\frac{I+B}{I} \cong \frac{B}{I\cap B}$
		\item $I, J \subseteq A$ ideali e $I \subseteq J$. Allora vale $\frac{\frac{A}{I}}{\frac{J}{I}} \cong \frac{A}{J}$ \\
			Si ha inoltre la corrispondenza tra gli ideali di $\frac{A}{I}$ e gli ideali $J \subseteq A$ tali che $I \subseteq J$. In questa corrispondenza i primi ed i massimali si corrispondono
		\item $IJ \subseteq I \cap J$. Se vale $I + J = 1$ allora $IJ = I \cap J$
		\item È FALSO che $I \cap (J + K) = (I \cap J) + (I \cap K)$. FALSO
		\item $I \subseteq \sqrt{I}$
		\item ($A$ dominio) $a$ primo $\implies$ $a$ irriducibile
		\item ($A$ UFD) $a$ irriducibile $\implies$ $a$ primo
		\item Se $H \subseteq A \times B$ è ideale allora $H = I \times J$ con $I \subseteq A$, $J \subseteq B$ ideali
		\item $A \cong A_1 \times A_2 \sse \exists e \in A, e \neq 0,1 \quad e^2 = e$
		\item $\cD(A) = \cup_{a \notin A^*} (0 : a) = \cup_{a \notin A^*} \sqrt{(0 : a)}$ e $\sqrt{\cD(A)} = \cD(A)$, anche se non è necessariamente un ideale
		\item $\{ E_\lambda \}_{\lambda \in \Lambda}$ sottoinsiemi di $A$. Allora $\cup_{\lambda \in \Lambda} \sqrt{E_\lambda} = \sqrt{\cup_{\lambda \in \Lambda} E_\lambda}$
		\item Sia $A$ dominio con un numero infinito di elementi e $\mid A^* \mid < \infty$ allora $A$ possiede infiniti ideali massimali
		\item $I$ massimale $\implies I$ primo $\implies I$ primario. Inoltre $A$ dominio $\sse (0)$ ideale primo
		\item Sono equivalenti:
			\begin{itemize}
				\item $A$ ha un unico ideale massimale
				\item $\exists \mathfrak{m} \subseteq A$ ideale massimale $\tc \forall a \in A \setminus \mathfrak{m} \implies a \notin A^*$
				\item $\exists \km \subseteq A$ ideale massimale $\tc$ ogni elemento della forma $1 + \km$ è invertibile
			\end{itemize}
		\item $a \in \cJ(A) \sse \forall b \in A \quad 1-ab \in A^*$
		\item $\sqrt{I} = \cap_{I \subseteq P \text{ primi }} P$
		\item ({\bf Lemma di Scansamento}) $P_1, \ldots, P_n$ ideali primi. Sia $I \subseteq A$ ideale $\tc I \subseteq \cup_{i=1}^n P_i$. Allora $\exists j \tc I \subseteq P_j$
		\item $I_1, \ldots, I_n$ ideali e $P$ ideale primo. $\cap_{i=1}^n I_i \subseteq P \implies \exists j \tc I_j \subset P$. Inoltre se $P = \cap_i I_i$ allora $\exists j \tc I_j = P$
		\item ({\bf Teorema cinese}) Siano $I_1, \ldots, I_n \subseteq A$ ideali tali che $I_i + I_j = 1$. Allora $\forall a_1, \ldots, a_n \in A \quad \exists a \in A \tc a \equiv a_i (I_i)$
		\item $A$ anello c.u. Allora si ha che
			\begin{itemize}
				\item $f \in A[x]$ è un'unità $\sse$ $f = \sum_{i=0}^n a_i x^i$ con $a_i \in A$ tali che $a_0 \in A^*$ e $a_i \in \cN(A) \quad \forall i \ge 1$
				\item $f \in A[x]$ è nilpotente $\sse$ $\forall i \quad a_i \in \cN(A)$
				\item $f \in A[x]$ è divisore di zero $\sse$ $\exists c \in A, c \neq 0 \tc cf = 0$
			\end{itemize}
			Si ha inoltre per gli anelli di polinomi che
			\begin{itemize}
				\item $I$ primo $\sse I[x]$ primo
				\item $I$ primario $\sse I[x]$ primario
			\end{itemize}
			NON è vero che tutti gli ideali di $A[x]$ sono del tipo $I[x]$, come ad esempio $(x)$
		\item Gli ideali primi di $\bbZ[x]$ sono dei seguenti tipi:
			\begin{itemize}
				\item $(0)$
				\item $(p)[x]$ con $p \in \bbP$
				\item $(f(x))$ con $f$ irriducibile
				\item $(p, f(x))$ con $p \in \bbP$ e $f$ irriducibile modulo $p$ (Questi sono anche massimali)
			\end{itemize}
		\item $u \in A^*$, $a \in \cN(A)$, allora $u + a \in A^*$ (Somma di un nilpotente e di un invertibile è invertibile)
		\item $I$ primo $\implies I $ irriducibile
		\item In $A[x]$ si ha $\cN(A[x]) = \cJ(A[x])$ (Mentre in generale vale solo che $\cN(A) \subseteq \cJ(A)$)
		\item Sia $\phi: A \rar B$ omomorfismo di anelli. Allora
			\begin{itemize}
				\item $\phi(\cN(A)) \subseteq \cN(B)$
				\item Se $\phi$ è surgettivo allora $\phi(\cJ(A)) \subseteq \cJ(B)$
				\item $A$ semilocale (con un numero finito di ideali massimali) $\implies \phi(\cJ(A)) = \cJ(B)$
			\end{itemize}
		\item $A$ PID $\implies \cJ(A) = \cN(A)$
		\item $A \tc$ ogni ideale è primo $\implies$ $A$ è un campo
		\item $A \tc$ ogni ideale primo è principale $\implies A$ è un anello ad ideali principali
		\item $\sqrt{I}$ massimale $\implies I$ primario.
		\item ({\bf Teorema della base di Hilbert}) Se $A$ è un anello Nötheriano, allora $A[x]$ è Nötheriano
	\end{itemize}
	
	\section*{Basi di Gröbner}
	\subsection*{Ideali Monomiali}
	Un ideale monomiale in $K[x_1, \ldots, x_n]$ è un ideale generato dai monomi
	\begin{itemize}
		\item ({\bf Criterio di appartenenza}) Sia $I$ un ideale monomiale e $f \in K[x_1, \ldots, x_n]$, $f = \sum_\beta c_\beta x^\beta$ con $c_\beta \in K$. Allora $f \in K \sse \forall \beta x^\beta \in I$
		\item ({\bf Lemma di Dickson}) Ogni ideale monomiale è finitamente generato. (La frontiera minimale di un ideale monomiale è unica, e viene detta Escalièr)
		\item ({\bf Operazioni con ideali monomiali}) Siano $I_1 = (m_1, \ldots, m_k)$ e $I_2 = (n_1, \ldots, n_s)$ con $m_i, n_j$ monomi. Allora si ha
			\begin{itemize}
				\item $I_1 + I_2 = (m_1, \ldots, m_k, n_1, \ldots, n_s)$
				\item $I_1 \cap I_2 = (\MCD_{i,j} (m_i, n_j))$
				\item $I_1 \cdot I_2 = (m_i \cdot n_j)_{i,j}$
				\item ({\bf Iatto}) $(I, m \cdot n) = (I, m) \cap (I, n)$ se $\MCD(m, n) = 1$ come monomi
				\item $I$ primo $\sse I = (x_{i_1}, \ldots, x_{i_k})$ (ed è massimale solo se le variabili compaiono tutte, ma DEVE essere monomiale)
				\item $I = \sqrt{I}$ (ovvero $I$ è radicale) $\sse \sqrt{m_i} = m_i \forall i$
				\item $I$ è primario $\sse I = (x_{i_1}^{\alpha_1}, \ldots, x_{i_k}^{\alpha_k}, m_1, \ldots, m_s)$ dove $m_1, \ldots, m_s \in K[x_{i_1}, \ldots, x_{i_k}]$
				\item $I$ è irriducibile $\sse$ $I = (x_{i_1}^{\alpha_1}, \ldots, x_{i_k}^{\alpha_k})$
				\item $I\cdot J = I \cap J \sse \forall i,j \quad \MCD(m_i, n_j) = 1$
				\item $I : J = \cap_i (I : n_i)$ e $I : (n_i) = (\frac{m_j}{\MCD(n_i, m_j)})_{j}$
			\end{itemize}
		\item Notare che usando la terza relazione del punto precedente possiamo spezzare ogni ideale monomiale in ideali primari e utilizzando $\sqrt{I \cap J} = \sqrt{I} \cap \sqrt{J}$ si possono calcolare anche gli ideali primi associati. \\
			Inoltre con la decomposizione in primari si calcolano bene i divisori di zero, i nilpotenti, etc.
	\end{itemize}
	
	\subsection*{Ordinamenti Monomiali Comuni}
	\begin{itemize}
		\item LEX $x_1 > x_2 > \ldots > x_n$. Dico che $\alpha \ge \beta \sse$ In $\alpha - \beta$ la prima coordinata $\neq 0$ è positiva
		\item DEGLEX Sia $\mid \alpha \mid := \sum_i \alpha_i$. Allora $\alpha \ge \beta \sse$ si ha $\mid \alpha \mid \ge \mid \beta \mid$ oppure $\mid \alpha \mid = \mid \beta \mid$ e vale $\alpha \ge \beta$ con LEX
		\item DEGREVLEX $\alpha \ge \beta \sse \mid \alpha \mid > \mid \beta \mid$ oppure si ha $\mid \alpha \mid = \mid \beta \mid$ e in $\alpha - \beta$ l'ultima coordinata $\neq 0$ è negativa
	\end{itemize}

	\subsection*{}
	\begin{itemize}
		\item ({\bf Algoritmo di Divisione}) Siano $f_1, \ldots, f_k, f \in K[x_1, \ldots, x_n]$ allora $\exists a_1, \ldots, x_k, r \in K[x_1, \ldots, x_n]$ tali che $f = \sum_i a_i f_i + r$ e $\Deg (a_i f_i) \le \Deg(f)$. Inoltre se $r = \sum_\alpha r_\alpha x^\alpha$ si ha che se $r_\alpha \neq 0$ allora $x^\alpha \in (\Lt(f_1), \ldots, \Lt(f_k))$ \\
		Notiamo che posso fare dei passaggi "a mano" prima di partire con l'algoritmo di divisione e lui funzionerà comunque. La cosa importante è ricordarsi di soddisfare la condizione $\Deg (a_i f_i) \le \Deg(f)$ ad ogni passaggio.
		\item ({\bf Base di Gröbner}) Un insieme di polinomi $g_1, \ldots, g_k$ generatori di un ideale $I$ i cui leading term generano $\Lt(I)$ si dicono base di Gröbner. Sono equivalenti inoltre:
			\begin{itemize}
				\item $\forall f \quad \exists ! r$ resto della divisione di $f$ per $\{g_1, \ldots, g_k\}$
				\item $\forall f \in I = (g_1, \ldots, g_k)$ si ha $r = 0$ dall'algoritmo di divisione
				\item $\forall i,j \quad S(g_i, g_j)$ ha resto $r = 0$ nell'algoritmo di divisione
			\end{itemize}
		Dove per divisione si intende un risultato che soddisfi le ipotesi dell'algoritmo di divisione
		\item ({\bf Base di Gröbner ridotta}) Una BdG $G = \{ g_1, \ldots, g_k \}$ si dice ridotta se è minimale per inclusione e inoltre
			\begin{itemize}
				\item $\Lc(g_i) = 1 \quad \forall i$
				\item $(\Deg(g_1), \ldots, \Deg(g_k))$ sono un'escalièr per $\Deg(I)$
				\item $\forall g_i \quad g_i = \sum_\alpha c_\alpha x^\alpha$ allora $x^\alpha \notin \Lt(G \setminus \{g_i\})$
			\end{itemize}
		Teorema: La base ridotta è unica. Per ridurre una BdG basta prendere ciascun elemento $g$ ed effettuare la divisione per $G \setminus \{g\}$
		\item ({\bf S-polinomio}) Dati $f, g \in K[x_1, \ldots, x_n]$ e supponiamo $f = c_\alpha x^\alpha + f_1$ e $g = d_\beta x^\beta + g_1$ con $\Deg f = \alpha, \Deg g = \beta$. Allora dico S-polinomio tra $f, g$ il polinomio definito da $\gamma = (\gamma_1, \ldots, \gamma_n)$ con $\gamma_i = \max(\alpha_i, \beta_i)$
			$$ S(f, g) = \frac{x^\gamma}{c_\alpha x^\alpha} f - \frac{x^\gamma}{d_\beta x^\beta} g $$
		\item ({\bf Eliminazione di LEX}) $I \subseteq K[x_1, \ldots, x_n]$ allora $I_k = I \cap K[x_{k+1}, \ldots, x_n]$ è il $k$-esimo ideale di eliminazione. Vale il teorema: Se $G$ è una BdG rispetto a LEX con $x_1 \ge \ldots \ge x_n$ allora $\forall k = 1, \ldots, n-1$ si ha che $G_k = G \cap K[x_{k+1}, \ldots, x_n]$ è BdG di $I_k$
		\item ({\bf Cose calcolabili}) Dati $I, J \subseteq K[x_1, \ldots, x_n]$ e note le loro due BdG si ha
			\begin{itemize}
				\item ({\bf Intersezione}) $I \cap J = (tI, (1-t)J) \cap K[x_1, \ldots, x_n]$ dove quindi bisognerà usare l'ordinamento LEX con $t$ come variabile più pesante per poter usare eliminazione
				\item ({\bf Colon}) Se $\BdG(J) = \{h_1, \ldots, h_r\}$ allora $I : J = \cap_{i=1}^r (I : h_i)$. \\
					Se ora ho $f \in K[x_1, \ldots, x_n]$ e voglio calcolare $I : (f) = \{g \mid gf \in I\}$ allora ho che $I : (f) = \frac{1}{f} \cdot (I\cap (f))$, ovvero se $\BdG(I \cap (f)) = \{g_1 f, \ldots, g_k f\}$ allora ho $\BdG(I : (f)) = \{g_1, \ldots, g_k\}$
				\item ({\bf Ker di morfismi}) Sia $\Phi: K[x_1, \ldots, x_n] \rar K[y_1, \ldots, y_n]$ tale che $f_i(Y) := \Phi(x_i)$. Allora si ha $\Ker \Phi = (x_1 - f_1(Y), \ldots, x_n - f_n(Y)) \cap K[x_1, \ldots, x_n]$ ovvero bisogna calcolare l'ideale di eliminazione senza le $Y$
			\end{itemize}
		\item ({\bf Sistemi di equazioni polinomiali}) Cerchiamo le soluzioni comuni di $f_1 = 0, \ldots, f_n = 0$ in $K^n$. Valgono:
			\begin{itemize}
				\item Se $K$ è algebricamente chiuso, il sistema non ha soluzioni se e solo se $1 \in I = (f_1, \ldots, f_n)$, che si vede subito se c'è o meno con una BdG
			\end{itemize}
	\end{itemize}
	
	\section*{Ideali e Varietà}
	Siano $I, J \subseteq K[x_1, \ldots, x_n]$ ideali e $V$ varietà affine. Allora vale
	\begin{itemize}
		\item $I \subseteq J \implies \cV(J) \subseteq \cV(I)$
		\item $I \subseteq \cI(\cV(I))$
		\item $\cV(\cI(V)) = V$
		\item $\cV(I) \subseteq \cV(J) \implies \cI(\cV(J)) \subseteq \cI(\cV(I))$
		\item $\cV(I + J) = \cV(I) \cap \cV(J)$
		\item $\cV(I\cdot J) = \cV(I) \cup \cV(J) = \cV(I \cap J)$
		\item $\cV(I) = \cV(\sqrt{I})$
	\end{itemize}
	Valgono inoltre i seguenti fatti:
	\begin{itemize}
		\item 
	\end{itemize}
	
\end{document}
