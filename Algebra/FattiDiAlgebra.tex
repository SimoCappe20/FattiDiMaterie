\documentclass[a4paper,NoNotes,GeneralMath]{stdmdoc}

\newcommand{\Ord}{\text{ord }}
\newcommand{\sgr}{\sqsubseteq}
\newcommand{\nrm}{\lhd}
\newcommand{\gen}[1]{\langle #1 \rangle}
\newcommand{\Int}{\text{Int }}
\newcommand{\Ind}{\text{i }}
\newcommand{\Aut}{\text{Aut }}
\newcommand{\MCD}{\text{MCD }}
\newcommand{\mcm}{\text{mcm }}
\newcommand{\isom}{\equiv}

\usepackage{pgf}
\usepackage{tikz}
\usetikzlibrary{arrows,automata}

\begin{document}
	\title{Teoria dei Gruppi}

	\section*{Enunciati}
	Nel seguito $G$ indica un qualsiasi gruppo, viene indicata con $e$ l'unità del gruppo. La notazione usata è quella moltiplicativa. $H \sgr G$ indica che $H$ è sottogruppo di $G$ (eventualmente coincidente). $H \nrm G$ indica che $H$ è un sottogruppo normale di $G$.
	\begin{itemize}
		\item Due qualsiasi laterali destri di $H \sgr G$ in $G$ ($Ha$ e $Hb$) sono in corrispondenza biunivoca attraverso la funzione $ah \mapsto bh$
		\item Esiste inoltre una corrispondenza biunivoca tra l'insieme dei laterali destri e quello dei laterali sinistri di uno stesso sottogruppo $H$
		\item ({\bf Teorema di Lagrange}) $G$ finito e $H \sgr G$, allora $\Ord H \mid \Ord G$
		\item $G$ finito, $a \in G$ allora $\Ord a \mid \Ord G$ e $a^{\Ord G} = e$
		\item ({\bf Ciclicità degli ordini primi}) $G$ finito con ordine primo ($\Ord G = p \in \bbP$), allora $G$ è ciclico
		\item ({\bf Sottogruppo prodotto}) $H, K \sgr G$. Allora $HK \sgr G \sse HK = KH$
		\item ({\bf Ordine del prodotto}) $H, K \sgr G$ con $H$ e $K$ sottogruppi finiti. Supponiamo che $HK \sgr G$. Allora $\Ord(HK) = \frac{\Ord(H)\Ord(K)}{\Ord(H\cap K)}$
		\item ({\bf Definizione di sottogruppo normale}) $N \nrm G \sse \forall x\in G \quad xHx^{-1} = H \sse \forall x \in G \quad xHx^{-1} \subseteq H \sse \forall x \in G \quad xH = Hx$
		\item ({\bf Gruppo quoziente}) Se $N \nrm G$, allora anche $G/N$ è un gruppo. Inoltre se $G$ è finito, vale $\Ord(G/N) = \frac{\Ord(G)}{\Ord(N)}$
		\item ({\bf Proiezione al quoziente}) $N \nrm G$. $\Phi: G \mapsto G/N$ definita da $\Phi(g) = Ng$ è un omomorfismo surgettivo.
		\item ({\bf Gruppi abeliani hanno tutti i sottogruppi normali}) $G$ abeliano. $N \sgr G \implies N \nrm G$.
		\item ({\bf Controimmagine di un normale è normale}) $N' \nrm G'$, $\Phi: G \rar G'$. Allora $\Phi^{-1}(N') \nrm G$.
		\item ({\bf Immagine di un normale con morfismo surgettivo è normale}) $N \nrm G$, $\Phi: G \rar G'$ omomorfismo sugettivo. Allora $\Phi(N) \nrm G'$.
		\item ({\bf Normalità del Ker}) $\Phi: G \mapsto H$ omomorfismo surgettivo. $K = \Ker \Phi \implies K \nrm G$
		\item ({\bf L'immagine è un sottogruppo}) $\Phi: G \rar G'$ omomorfismo. $\Img \Phi \sgr G'$ (ma NON è detto che sia normale)
		\item ({\bf Immagini inverse}) $\Phi: G \mapsto H$ omomorfismo. $\Ker \Phi = K \implies \Phi^{-1}\Phi(x) = Kx$
		\item ({\bf Primo teorema di Omomorfismo}) $\Phi: G \mapsto H$ omomorfismo surgettivo con $K = \Ker \Phi$. Allora $G/K \cong H$
		\item ({\bf Variante del Primo teorema di Omomorfismo}) $f: G \mapsto G'$ omomorfismo surgettivo. $H \nrm G, H \sgr K, K = \Ker f$. Allora $\exists ! \phi: \frac{G}{H} \rar G'$ non necessariamente iniettivo tale che $f = \phi \circ \pi_{\frac{G}{H}}$
		\item ({\bf "Inversi del teorema di Lagrange"}) Se $G$ è ciclico, $\Ord G = n$ si ha $\forall d \mid n \quad \exists ! H \sgr G \tc \Ord H = d$. \\ Se $G$ è abeliano, $\Ord G = n$ si ha $\forall d \mid n \quad \exists H \sgr G \tc \Ord H = d$ ma in generale non è unico.
		\item ({\bf Condizione equivalente al prodotto diretto}) $G \isom H \times K$ $\sse$ $\exists H, K \nrm G \tc H\cap K = (e), HK = G$
		\item ({\bf Teorema di Cauchy}) Sia $p \in \bbP \tc p \mid \Ord G$. Esiste allora $a \neq e \tc a^p = e$
		\item ({\bf Primo teorema di Sylow}) Sia $p \in \bbP \tc p^\alpha \mid \Ord G, p^{\alpha + 1} \nmid \Ord G$. Allora $G$ ha un sottogruppo di ordine $p^\alpha$. Inoltre se $G$ è abeliano tale sottogruppo è unico.
		\item ({\bf Secondo teorema di Syolow}) Sia $G$ un gruppo finito. Allora tutti i $p$-Sylow sono coniugati.
		\item ({\bf Corollario}) Dato un gruppo finito $G$, il numero dei $p$-Sylow di $G$ è uguale a $i_{G}(N_{G}(P))$, dove $P$ è un qualsiasi $p$-Sylow di $G$. In particolare, un $p$-Syolow $P$ è normale sse non ci sono altri $p$-Sylow oltre a $P$.
		\item ({\bf Terzo teorema di Sylow}) Detto $n_{p}$ il numero dei $p$-Sylow di un gruppo finito $G$, valgono $n_p \equiv 1$ (mod $p$) e $n_p \mid \Ord G$.   
		\item ({\bf Corrispondenza tra gruppi normali}) Sia $\Phi: G \mapsto G'$ omomorfismo surgettivo. $K = \Ker \Phi$. Dato $H' \sgr G'$ si definisca $H = \{x\in G \mid \Phi(x) \in H'\}$. \\ Si ha che $H \sgr G \tc K \subseteq H$. Inoltre se $H' \nrm G'$ allora $H \nrm G$. L'associare $H'$ ad $H$ stabilisce una corrispondenza biunivoca dell'insieme di tutti i sottogruppi di $G'$ sull'insieme di tutti i sottogruppi di $G$ che contengono $K$
		\item ({\bf Secondo teorema di Omomorfismo}) $\Phi: G \mapsto G'$ omomorfismo surgettivo, $K = \Ker \Phi$. Si prenda ora $N' \nrm G'$ e sia $N = \{x \in G \mid \Phi(x) \in N'\}$. Allora $G/N \cong G'/N'$ oppure, in modo equivalente, $G/N \cong (G/K)/(N/K)$.
		\item ({\bf Il centro è un sottogruppo normale}) $Z(G) \nrm G$.
		\item ({\bf Caratterizzazione degli automorfismi interni}) $\Int G \cong G/Z$ con $Z = Z(G)$ centro di $G$. Inoltre $\Int G \nrm \Aut G$.
		\item ({\bf Teorema di Cayley}) Ogni gruppo è isomorfo ad un sottogruppo di $S(X)$, per un opportuno $X$.
		\item ({\bf Teorema X}) Se $G$ è un gruppo, $H \sgr G$, $X$ l'insieme di tutti i laterali destri di $H$ in $G$, esiste un omomorfismo $\Phi: G \rar S(X)$. Inoltre $\Ker \Phi$ è il più grande sottogruppo normale di $G$ contenuto in $H$.
		\item ({\bf Corollario dell'indice fattoriale}) Se $G$ è un gruppo finito e $H \neq G$ un sottogruppo di $G$ tale che $\Ord(G) \nmid \Ind_G(H)!$, allora $H$ deve contenere un sottogruppo normale non banale di $G$. In particolare, $G$ non può essere semplice.
		\item ({\bf Argomento di Frattini}) Sia $G$ un gruppo finito e $H \nrm G$; sia $P$ un $p$-Sylow di $H$. Allora $G=H N_{G}(P)$.
		\item ({\bf Corollario}) Dato un $p$-Sylow $P \sgr G$ vale $N_{G}(N_{G}(P)) = N_{G}(P)$.
	\end{itemize}

	\section*{Particolari tipi di gruppi}
	\begin{itemize}
		\item ({\bf I gruppi ciclici sono abeliani}) $G$ ciclico $\implies G$ abeliano. (Segue dall'associatività dell'operazione di gruppo)
		\item ({\bf Ciclicità dei gruppi con ordine primo}) $G$ gruppo. $\Ord G = p \in \bbP \implies G$ è ciclico. (Basta usare Cauchy)
		\item ({\bf Esiste un unico gruppo ciclico di ogni ordine}) $G$ gruppo ciclico. $\Ord G = n \implies G \isom \bbZ_{n}$
		\item ({\bf Abelianità di Gruppo con quoziente sul centro ciclico}) $G$ gruppo. $G/Z(G)$ ciclico $\implies G$ abeliano
	\end{itemize}

	\section*{Controesempi}
	\begin{itemize}
		\item ({\bf Gruppo non abeliano con tutti i sottogruppi normali}) $Q_8 = \{1, i, j, k, -1, -i, -j, -k\}$ con le regole di moltiplicazione tra quaternioni. ($i^2 = j^2 = k^2 = 1$, $ij = k, ji = -k, \ldots$)
	\end{itemize}

	\section*{Trucchi vari}
	\begin{itemize}
		\item Il modo più utile di usare l'informazione $\MCD(a, b) = 1$ è tramite Bèzout: $\exists s, t \tc \quad as + bt = 1$, soprattutto se $a$ e $b$ sono ordini di gruppi.
		\item Se $N \nrm G$, $x^{\Ind_G(N)} \in N$ (poiché $\Ind_G(N)$ è l'ordine del gruppo quoziente $G/N$)
		\item Se $G^k \sgr G$, allora $G^k \nrm G$. (Segue banalmente da $ga^k g^{-1} = (gag^{-1})^k$)
		\item Se $H \sgr G$, $\Ord(H) > \frac{\Ord(G)}{2} \implies H = G$
	\end{itemize}

	\section*{Gruppi Ciclici}
	\begin{itemize}
		\item $H, K \sgr G$, $\Ord(H) = a, \Ord(K) = b$. Se $\MCD(a,b) =1$, allora $H \cap K = (e)$. Infatti $H \cap K \sgr H, H\cap K \sgr K \implies \Ord(H\cap K) \mid \Ord(H), \Ord(H\cap K) \mid \Ord(K) \implies \Ord(H \cap K) = 1$.
		\item Se $H \cap K = (e)$ e $H, K \sgr G$ con $G$ abeliano si ha: Siano $h \in H, k \in K$, $\Ord(h) = r, \Ord(k) = s$. Allora $\Ord(hk) = \mcm(r,s)$. (Infatti $(hk)^{\mcm(r,s)} = h^{\mcm(r,s)} k^{\mcm(r,s)} = e e = e$. Inoltre supponiamo $\exists t < \mcm(r,s) \tc (hk)^t = e$ Allora $h^t k^t = e \implies h^t = k^{-t} \in H \cap K \implies h^t = k^{-t} = e \implies r\mid t, s \mid t \implies \mcm(r,s) \mid t$)
	\end{itemize}

	\section*{Caratteristiche di $S_n$}
	\begin{itemize}
		\item $S_n$ NON è abeliano per $n \ge 3$. Infatti $(1 2)$ e $(1 3)$ non commutano
		\item $S_n$ ha come sottogruppo normale di indice 2 il gruppo alterno $A_n$ formato dalle permutazioni pari
		\item Il centro di $S_n$ è banale per $n \ge 3$. Per questo motivo $S_n$ NON è nilpotente per $n \ge 3$
		\item $S_n$ per $n\neq 2,6$ è un gruppo completo poiché non ha centro ed ogni automorfismo è interno
	\end{itemize}

	\section*{Layout completo di $S_4$}
	$S_4$ è il gruppo delle permutazioni di quattro elementi. $A_4$ è il gruppo delle permutazioni pari. $V_4$ è il gruppo dei prodotti di 2-cicli disgiunti ($V_4 = \{(), (12)(34), (13)(24), (14)(23)\}$). $D_8$ è il gruppo diedrale di ordine otto.

	$S_4$ contiene le seguenti permutazioni:
	\begin{itemize}
		\item $1$ identità: $()$
		\item $6$ 2-cicli: $(12), (13), (14), (23), (24), (34)$
		\item $3$ prodotti di 2-cicli: $(12)(34), (13)(24), (14)(23)$
		\item $8$ 3-cicli: $(123), (124), (132), (134), (142), (143), (234), (243)$
		\item $6$ 4-cicli: $(1234), (1243), (1324), (1342), (1423), (1432)$
	\end{itemize}

	Altre caratteristiche di $S_4$:
	\begin{itemize}
		\item Abbiamo che $S_4$ è risolubile considerando la catena $(e) \subseteq V_4 \subseteq A_4 \subseteq S_4$
		\item $A_4 \nrm S_4$ (Poiché ha indice $2$)
		\item $V_4 \nrm S_4$ (conti)
		\item $D_8 \sgr S_4$ (prendendo $D_8 = \{(), (1234), (13)(24), (1432), (12)(34), (14)(23), (13)(24)\}$)
	\end{itemize}

	\section*{Gruppi diedrali $D_n$}
	\begin{itemize}
		\item ({\bf Presentazione}) $D_n = \lbrace s, r \mid s^2 = r^n = e, srs^{-1} = r^{-1} \rbrace$
		\item ({\bf Moltiplicazione}) $r^{i}s^{j} \cdot r^{a}s^{b} = r^{i + (-1)^{j} a} s^{j + b}$
		\item ({\bf Sottogruppi di $D_n$}) Si hanno i seguenti sottogruppi: Se $m \mid n$ si ha $C_m = \lbrace r^{\frac{n}{m}} \rbrace \nrm D_n$, $D_m = \lbrace r^{\frac{n}{m}}, sr^{k} \rbrace$ con $k = 0, 1, \ldots, \frac{n}{m} - 1$
		\item ({\bf Classi di coniugio di $D_n$, $n$ pari}) Sono $\{e\}$, $\{r^k, r^{-k}\} \quad \forall k \in \{1, \ldots, \frac{n}{2}\}$, $\{s, sr^2, \ldots, sr^{\frac{n}{2}}\}$, $\{sr, sr^3, \ldots, sr^{\frac{n}{2}-1}\}$
		\item ({\bf Classi di coniugio di $D_n$, $n$ dispari}) Sono $\{e\}$, $\{r^k, r^{-k}\} \quad \forall k \in \{1, \ldots, \frac{n-1}{2}\}$, $\{s, sr, sr^2, \ldots, sr^{n-1}\}$
		\item ({\bf Sottogruppi Normali di $D_n$}) $C_m \nrm D_n$, Se $n$ dispari allora nessun altro (tranne quelli banali), se $n$ pari si hanno i due sottogruppi $D_{\frac{n}{2}} \nrm D_n$
		\item ({\bf Sottogruppi Abeliani di $D_n$}) Tutti i $C_m$ e i $D_1, D_2$
	\end{itemize}

	\section*{Elenco dei gruppi di ordine piccolo}
	\noindent \begin{tabular}{lcc}
	{\bf Ordine} & {\bf Gruppi Abeliani} & {\bf Gruppi Non Abeliani} \\
	1 & $C_1$ & \\
	2 & $C_2$ & \\
	3 & $C_3$ & \\
	4 & $C_4$, $C_2 \times C_2$ & \\
	5 & $C_5$ & \\
	6 & $C_6$ & $S_3$ \\
	7 & $C_7$ & \\
	8 & $C_8$, $C_4 \times C_2$, $C_2 \times C_2 \times C_2$ & $D_4$, $Q_8$ \\
	9 & $C_9$, $C_3 \times C_3$ & \\
	\end{tabular} \vskip 1em

	\title{Teoria degli Anelli}
	\section*{Definizioni}
	\begin{itemize}
		\item ({\bf Ideale primo in un anello commutativo}) Se $A$ è un anello, allora si dice che l'ideale $P$ di $A$ è primo se: $P \subsetneq A$ e se $a, b \in A \tc ab \in P \implies a \in P \text{ oppure } b \in P$
		\item ({\bf Ideale massimale})
	\end{itemize}

	\section*{Proprietà degli ideali primi}
	\begin{itemize}
		\item Un ideale $I$ dell'anello commutativo $A$ è primo se e solo se l'anello quoziente $\frac{A}{I}$ è un dominio di integrità
		\item Un ideale $I$ di un anello $A$ è primo se e solo se $A \setminus I$ è chiuso rispetto alla moltiplicazione
		\item In un anello commutativo unitario ogni ideale massimale è anche un ideale primo
		\item ({\bf Lemma di Krull}) Ogni anello commutativo unitario ha almeno un ideale massimale (si può dimostrare usando il lemma di Zorn)
		\item Un anello commutativo è un dominio di integrità se e solo se $\{0\}$ è un ideale primo
		\item La controimmagine di un ideale primo attraverso un omomorfismo di anelli è un ideale primoh
	\end{itemize}
\end{document}
