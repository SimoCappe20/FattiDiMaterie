%\documentclass[a4paper,NoNotes,GeneralMath]{stdmdoc} 
\documentclass[a4paper,10pt,oneside]{math_article}

\title{Lezioni di Algebra 1}
\author{Persone varie ed eventuali}
\date{}

\begin{document}

\maketitle

\cleardoublepage	
  \section{Lezione 1}
    \subsection{Richiami sui gruppi}
      Iniziamo con un ripasso del corso di Aritmetica. Si dice \myname{gruppo} una coppia $(G,\cdot)$ con $e \in G$ elemento neutro, e in cui esiste sempre l'inverso. Un gruppo può essere abeliano se $\forall x,y\in G$ vale $xy=yx$. Di non abeliano conosciamo solo il gruppo delle permutazioni $S(n)$.
      
      Un gruppo si dice \myname{ciclico} se è generato da un solo elemento, i.e. $G = \gen x = \{x^n: n\in \mathbb Z\}$. Si dimostra facilmente che un gruppo ciclico è isomorfo a $\mathbb Z/n\mathbb Z$.
      
      Un gruppo ciclico è ovviamente abeliano ma NON vale il viceversa.
      
      Si dice insieme dei generatori di un gruppo G un insieme $S\subseteq G$ tale che non esistono $H<G$ con $S\subseteq H$, e si indica $\gen S =G$ ($S$ genera $G$). In generale $\gen S$ è l'insieme dei prodotti finiti di elementi di $S$ e dei loro inversi.
      
      \begin{mytheorem}[Teorema di Lagrange]
	Sia $G$ un gruppo con $\abs G = n$, sia $H<G$, con $\abs H = d$. Allora $d\mid n$.
      \end{mytheorem}
      
      \begin{mytheorem}[Teorema di Cauchy]
	Sia $G$ un gruppo con $\abs G = n$ e sia $d\mid n$. Inoltre supponiamo che sia vera almeno una tra
	\begin{itemize}
	  \item $G$ è ciclico;
	  \item $G$ è abeliano;
	  \item $d$ è primo.
	\end{itemize}
	
	Allora esiste $H<G$ con $\abs H=d$.
      \end{mytheorem}
      
      Dato $x\in G, H<G$ indichiamo con $xH=\{xh: h \in H\}$ la classe laterale sinistra(?) di $H$ rispetto a $x$. Analogalmente per la classe laterale destra. 
      
      \begin{mydef}[Sottogruppo normale]
	Si dice sottogruppo normale di un gruppo $G$ un sottogruppo $H$ tale che $xH=Hx\; \forall x\in G$, o equivalentemente $xHx^{-1}\subseteq H$. Si indica con $H \subnorm G$
      \end{mydef}
      
      Se un sottogruppo è normale, le classi laterali sinistre coincidono con quelle destre.
      
      La condizione di normalità di un sottogruppo è particolarmente importante nella definizione di quoziente: se $H\subnorm H$ posso definire una struttura su $G/H$ (i.e. l'insieme delle classi laterali di $H$) con $xH \cdot yH = xyH$.
      
      \begin{mydef}[Omomorfismo tra gruppi]
	Si dice omomorfismo di gruppi una funzione $f: G \rightarrow G'$ che conserva la struttura, i.e $f(x)f(y)=f(xy)$.
      \end{mydef}
      
      Vale il seguente fatto simpatico.
      
      \begin{myprop}
	I sottogruppi normali sono tutti e soli i $\Ker$ di omomorfismi.
      \end{myprop}
      \begin{proof}
	Ogni sottogruppo normale è $\Ker$ della proiezione canonica su $G/H$. Inoltre se $H = \Ker f$, allora $f(xh\inv{x})= f(x)f(h)f(\inv x)=e$ quindi $xh\inv x\in H$.
      \end{proof}
      
      Siamo pronti per riprendere il primo teorema di omomorfismo.
      \begin{mytheorem}\label{th:Hom1}
	Sia $f: G\rightarrow G'$ omomorfismo di gruppi, e sia $K= \Ker f$. Detta $\pi: G\rightarrow G/K$ la proiezione canonica di $G$ su $K$, allora 
	\[\exists!\phi: G/K \rightarrow G' \quad\mbox{tale che}\quad f=\phi\circ\pi\]
	
	Inoltre $\phi$ è iniettivo.
      \end{mytheorem}
      
      Sostituendo $K$ con un qualsiasi $H\subnorm \Ker f$ si ottiene lo stesso risultato perdendo però l'iniettività di $\phi$.
      
      \begin{myprop}
	Sia $f: G \rightarrow G'$ omomorfismo suriettivo, allora $H<G \Rightarrow f(H)<G'$, e viceversa $H'<G' \Rightarrow \inv f(H)<G$. In altre parole la relazione di essere sottogruppo si conserva per passaggio all'immagine e alla controimmagine. Lo stesso vale per sottogruppi normali.
      \end{myprop}

      Da questa proposizione e dal teorema \ref{th:Hom1} segue che i sottogruppi di $G$ che contengono $\Ker f$ sono in corrispondenza biunivoca con i sottogruppi di $G'$.
      
      \subsection{Scomposizione di un gruppo come prodotto}

      Dati due gruppi $G$ e $H$, sappiamo definire il prodotto $G \times H$ con l'operazione $(g,h)\cdot (g',h')=(gg',hh')$. Vogliamo ora, dato un gruppo $G$, trovare due sottogruppi $G_1$ e $G_2$ tali che $G=G_1\times G_2$. Ci viene in aiuto il seguente teorema:
      
      \begin{mytheorem}
	Sia $G$ un gruppo (non necessariamente finito) e siano $H\subnorm G, K \subnorm G$ tali che:
	\begin{itemize}
	 \item $H\cap K=\{e\}$;
	 \item $G = HK = \{hk: h\in H, g \in G\}$
	\end{itemize}
	Allora $G \isom H\times K$
      \end{mytheorem}
      \begin{proof}
	Definiamo la funzione $f: H\times K \rightarrow G$ tale che $f(h,k)=hk$. Per ipotesi sappiamo che è suriettiva; inoltre ha $\Ker$ banale, infatti 
	\[f(h,k)=hk=e \Rightarrow H \ni h = \inv k \in K \Rightarrow h,k \in H \cap K = \{e\} \]
	ed è quindi iniettiva.
	
	Verifichiamo che è un omomorfismo: 
	\begin{gather*}
	  f(h,k)f(h',k')=hkh'k' \\
	  f\left((h,k)\cdot(h',k')\right)=f(hh',kk')=hh'kk'
	\end{gather*}
      
	Quindi se gli elementi di $H$ commutassero con quelli di $K$ avrei finito, infatti avrei costruito un isomorfismo tra $H\times K$ e $G$.
	
	Dimostriamo quindi che $hk=kh$ per ogni $ h\in H, k \in K$. Questo equivale a dimostare $hk\inv h\inv k=e$. Ma per la normalità (si dirà così?) di $H,K$ abbiamo
	\[
	H \ni h\underbrace{k\inv h \inv k}_{\in H} = \underbrace{hk\inv h}_{\in K}\inv k \in K
	\]
	e visto che $H\cap K=\{e\}$ ho finito.
      \end{proof}
      
      Un esempio può essere $(\mathbb C^\star, \cdot)$ che può essere scomposto in $R^+$ per la circonferenza unitaria $\{z:\abs z=1\}$, ritrovando così l'usale scomposizione di un complesso come $(\rho,\theta)$.
      
      

      
\end{document}
