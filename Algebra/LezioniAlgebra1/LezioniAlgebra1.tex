%\documentclass[a4paper,NoNotes,GeneralMath]{stdmdoc}
\documentclass[a4paper,10pt,oneside]{article}
\usepackage[utf8x]{inputenc}
\usepackage{amsmath}
\usepackage{amsthm}
\usepackage{amssymb}
\usepackage{hhline}
\usepackage{amsfonts}
\usepackage[italian]{babel}
\usepackage{hyphenat}
\usepackage[linktoc=all]{hyperref}

%Comando per la restrizione, servirà? Boh

\newcommand\restr[2]{{% we make the whole thing an ordinary symbol
  \left.\kern-\nulldelimiterspace % automatically resize the bar with \right
  #1 % the function
  \vphantom{\big|} % pretend it's a little taller at normal size
  \right|_{#2} % this is the delimiter
  }}
  
%Ricordarsi di cambiare la lingua e mettere la sillabazione

%Per dichiarare un operatore da stampare dritto (e.g. limsup) usare
%\DeclareMathOperator{\comando}{cosa si deve stampare}

%Usate \myname per evidenziare i nomi nuovi! Funziona come \emph, ma si può cambiare se non ci garba.
\newcommand{\myname}[1]{\emph{#1}} 

%Un po' di scorcatoie random
\newcommand{\nin}{\not\in}
\newcommand{\parti}[1]{\mathcal{P}(#1)}
\newcommand{\rel}{\mathcal R}
\newcommand{\abs}[1]{\left|#1\right|} %Valore assoluto
\newcommand{\gen}[1]{\left\langle#1\right\rangle} %Gruppo generato da x
\newcommand{\isom}{\cong}

%TEOREMI IN CORSIVO
\theoremstyle{plain}
\newtheorem{mytheorem}{Teorema}[section]
\newtheorem{mydef}[mytheorem]{Definizione}
\newtheorem{mylemma}[mytheorem]{Lemma}
\newtheorem{myprop}[mytheorem]{Proposizione}

\newtheorem{myax}[mytheorem]{Assioma}


%TEOREMI NON IN CORSIVO
\theoremstyle{definition}
\newtheorem{myex}{Esercizio}
\newtheorem{mycor}[mytheorem]{Corollario}
%Note
\theoremstyle{remark}
\newtheorem*{myobs}{Osservazione}


\setcounter{tocdepth}{4}


%\setlength{\parskip}{\baselineskip}%
\setlength{\parindent}{0pt}%

\title{Lezioni di Algebra 1}
\author{Persone varie ed eventuali}
\date{}


\begin{document}


\maketitle

\cleardoublepage	
  \section{Lezione 1}
    \subsection{Richiami sui gruppi}
      Iniziamo con un ripasso del corso di Aritmetica. Si dice \myname{gruppo} una coppia $(G,\cdot)$ con $e \in G$ elemento neutro, e in cui esiste sempre l'inverso. Un gruppo può essere abeliano se $\forall x,y\in G$ vale $xy=yx$. Di non abeliano conosciamo solo il gruppo delle permutazioni $S(n)$.
      
      Un gruppo si dice \myname{ciclico} se è generato da un solo elemento, i.e. $G = \gen x = \{x^n: n\in \mathbb Z\}$. Si dimostra facilmente che un gruppo ciclico è isomorfo a $\mathbb Z/n\mathbb Z$.
      
      Un gruppo ciclico è ovviamente abeliano ma NON vale il viceversa.
      
      Si dice insieme dei generatori di un gruppo G un insieme $S\subseteq G$ tale che non esistono $H<G$ con $S\subseteq H$, e si indica $\gen S =G$ ($S$ genera $G$). In generale $\gen S$ è l'insieme dei prodotti finiti di elementi di $S$ e dei loro inversi.
      
      \begin{mytheorem}[Teorema di Lagrange]
	Sia $G$ un gruppo con $\abs G = n$, sia $H<G$, con $\abs H = d$. Allora $d\mid n$.
      \end{mytheorem}
      
      \begin{mytheorem}[Teorema di Cauchy]
	Sia $G$ un gruppo con $\abs G = n$ e sia $d\mid n$. Inoltre supponiamo che sia vera almeno una tra
	\begin{itemize}
	  \item $G$ è ciclico;
	  \item $G$ è abeliano;
	  \item $d$ è primo.
	\end{itemize}
	
	Allora esiste $H<G$ con $\abs H=d\isom$.
      \end{mytheorem}
      
      
\end{document}
