\documentclass[a4paper,NoNotes,GeneralMath]{stdmdoc}
\usepackage{tikz}
\usetikzlibrary{arrows,automata}
\newcommand{\aperto}{\text{ aperto }}
\newcommand{\apertoin}{\text{ aperto in }}
\newcommand{\chiuso}{\text{ chiuso }}
\newcommand{\chiusoin}{\text{ chiuso in }}
\newcommand{\compatto}{\text{ compatto }}
\newcommand{\continua}{\text{ continua }}
\newcommand{\se}{\text{ se }}
\newcommand{\iin}{\text{ in }}
\newcommand{\intornodi}{\text{ intorno di }}
\newcommand{\allora}{\text{ allora }}
\newcommand{\Finiti}{Finiti}
\newcommand{\Arbitrari}{Arbitrari}
\newcommand{\Numerabili}{Numerabili}
\newcommand{\rd}[1]{{\color{red} #1 }}

\begin{document}
	\title{Topologia Generale}
	
	%<*DefinizioniProprieta>
	\section*{Definizioni delle Proprietà}
	\begin{itemize}
		\item[N1] Ogni punto ha una base di intorni numerabile.
		\item[N2] Lo spazio ha una base di aperti numerabile.
		\item[Sep] Se esiste un sottoinsieme denso e numerabile.
		\item[T0] Dati due punti c'è un aperto che li distingue. ($\forall x, y \in X \quad \exists A \aperto \tc x \in A, y \notin A \opp x \notin A, y \in A$)
		\item[T1] I punti sono chiusi.
		\item[T2] Punti distinti hanno intorni disgiunti.
		\item[Reg] Un punto ed un chiuso che non lo contiene hanno intorni disgiunti.
		\item[Norm] Ogni coppia di chiusi disgiunti ha intorni disgiunti.
		\item[T3] Reg + T0.
		\item[T4] Norm + T1.
		\item[Cpt] Ogni ricoprimento di aperti ha un sottoricoprimento finito.
		\item[Lind] Ogni ricoprimento di aperti ha un raffinamento numerabile.
		\item[Conn] Non esistono due aperti propri la cui unione è lo spazio intero. (Vale anche con i chiusi)
		\item[PathConn] Presi due punti esiste un arco che li connette ($\forall x, y \in X \quad \exists \gamma: [0,1] \rar X \continua \tc \gamma(0) = x, \gamma(1) = y$)
		\item[LocConn] Ogni punto ha una base di intorni connessi.
		\item[LocPathConn] Ogni punto ha una base di intorni connessi per archi.
		\item[LocCpt] Ogni punto ha una base di intorni compatti.
		\item[ParaCpt] Ogni ricoprimento aperto ha un raffinamento localmente finito.
		\item[Metr] Metrizzabile, ovvero esiste una distanza che induce la topologia.
	\end{itemize} \vskip 0.5cm
	%</DefinizioniProprieta>

	Da aggiungere: \\
	Metr: Sep <=> N2 <=> Lind, Metr => N1, Metr: Cpt => Sep, N2: Cpt <=> SeqCpt, Cpt => ParaCpt, ParaCpt + T2 => Norm.
	\begin{center}
		\begin{tikzpicture}[scale=2.5]
			%\node (ED) at (0,0) {ED};
			%\draw[->] (+1) -- node[above] {+ ID} (UFD);
			\node (T4) at (0,0) {T4};
			\node (T3) at (1,0) {T3};
			\node (T2) at (2,0) {T2};
			\node (T1) at (3,0) {T1};
			\node (T0) at (4,0) {T0};
			\node (N2) at (0,-2) {N2};
			\node (Lind) at (1,-1) {Lind};
			\node (Sep) at (1,-2) {Sep};
			\node (N1) at (1,-3) {N1};
			\node (PathConn) at (2,-1) {PathConn};
			\node (Conn) at (4, -1) {Conn};
			\node (Reg) at (2,-2) {Reg};
			\node (Norm) at (4,-2) {Norm};

			\draw[->] (T4) -- (T3);
			\draw[->] (T3) -- (T2);
			\draw[->] (T2) -- (T1);
			\draw[->] (T1) -- (T0);
			\draw[->] (N2) -- (Lind);
			\draw[->] (N2) -- (Sep);
			\draw[->] (N2) -- (N1);
			\draw[->] (PathConn) -- (Conn);
			\draw[->] (Reg) -- node[above] {+ N2 / + Lind} (Norm);
		\end{tikzpicture}
	\end{center} \vskip 0.5cm

	\section*{Equivalenze}
	\begin{itemize}
		\item ({\bf Condizione equivalente per essere una base}) Dato $X$ insieme e $\cB \subseteq \cP (X)$ esiste una topologia su $X$ di cui $\cB$ è una base se e soltanto se sono soddisfatte le seguenti due condizioni: $X = \cap \{B \mid B \in \cB \}$ e per ogni coppia $A,B \in \cB$ e per ogni punto $x \in A \cap B$ esiste $C \in \cB$ tale che $x \in C \subseteq A \cap B$.
		\item ({\bf Condizioni equivalenti alla continuità}) $f$ è continua $\sse$ controimmagine di aperti è aperta $\sse$ $\forall A \subseteq X \quad f(\bar{A}) \subseteq \bar{f(A)}$ $\sse$ $\forall x \in X \quad \forall U \tc f(x) \in U \quad \exists V \tc x \in V \quad f(V) \subseteq U$.
		\item ({\bf Condizioni equivalenti ad essere un omeomorfismo}) $f: X \rar Y$ continua. Allora $f$ è un omeomorfismo $\sse$ $f$ è chiusa e biggettiva $\sse$ $f$ è aperta e biggettiva.
		\item ({\bf Condizioni che implicano essere immersione}) Sia $f: X \rar Y$ continua. Allora se $f$ è chiusa ed iniettiva, essa è un'immersione chiusa. Se invece $f$ è aperta ed iniettiva, allora è un'immersione aperta.
		\item ({\bf Condizioni equivalenti alla sconnessione}) $X$ è sconnesso $\sse$ $X$ è unione disgiunta di due aperti propri $\sse$ $X$ è unione disgiunta di due chiusi propri.
	\end{itemize}
	
	\section*{Connessione}
	\begin{itemize}
		\item ({\bf Multilemma sulla connessione}) Sia $Y$ connesso e $f: X \rar Y$ una funzione {\it continua} (?) e surgettiva tale che $f^{-1}(y)$ è connesso $\forall y \in Y$. Se $f$ è aperta oppure se $f$ è chiusa, allora anche $X$ è connesso.
		\item ({\bf Connessione della chiusura}) Sia $Y$ un sottospazio connesso di $X$, e sia $Y \subseteq W \subseteq \bar{Y}$. Allora anche $W$ è connesso.
		\item ({\bf Chiusura delle componenti connesse}) Le componenti connesse sono chiuse.
		\item ({\bf Estensione delle componenti connesse}) Supponiamo di avere $\{Z_\lambda\}_{\lambda \in \Lambda} \tc Z_i$ è connesso $\forall i$ e tali che $\forall i, j \in \Lambda \quad \exists i=k_1, k_2, \ldots, k_n = j \in \Lambda$ tali che $Z_{k_l} \cap Z_{k_{l+1}} \neq \emptyset$. Allora $\cup_{\lambda \in \Lambda} Z_\lambda$ è connesso.
	\end{itemize}

	\section*{Compattezza}
	\begin{itemize}
		\item ({\bf Heine-Borel}) Un sottospazio $K \subset \bbR^n$ è compatto se e solo se è chiuso e limitato.
		\item ({\bf Multilemma sulla compattezza}) Sia $Y$ compatto e $f: X \rar Y$ una funzione chiusa. Se $f^{-1}(y)$ è compatto $\forall y \in Y$, allora anche $X$ è compatto.
		\item ({\bf Catene discendenti di compatti}) Siano $K_i$ chiusi e compatti tali che $\ldots \subset K_2 \subset K_1$ una catena discendente numerabile di chiusi non vuoti e compatti di uno spazio topologico. Allora $\cap_i K_i \neq \emptyset$.
		\item ({\bf Lemma di Wallace}) $X,Y$ spazi topologici. $A \subseteq X, B \subseteq Y$ sottospazi compatti e $W \subset X \times Y$ un aperto tale che $A \times B \subseteq W$. Allora $\exists U \subseteq X, V \subseteq Y$, aperti tali che $A \subseteq U, B \subseteq V, U\times V \subseteq W$.
		\item ({\bf Compatti hanno proiezioni chiuse}) Se $X$ è compatto, la proiezione $p: X \times Y \rar Y$ è un'applicazione chiusa.
		\item ({\bf Localmente compatto $\implies$ ammette un ricoprimento fondamentale in compatti}).
	\end{itemize}

	\section*{Compattificazioni}
	\begin{itemize}
		\item ({\bf La compattificazione di Alexandroff è $T_2$}) $\hat{X}$ è di Hausdorff se e solo se $X$ è di Hausdorff ed ogni punto di $X$ possiede un intorno compatto.
		\item ({\bf Immersioni aperte si estendono ad Alexandroff}) $f: X \rar Y$ immersione aperta. Allora l'applicazione $g: Y \rar \hat{X}$ definita da $g(y) := \left\{ \begin{array}{cr} x & \text{ se } y = f(x) \\ \infty & \text{ se } y \notin f(X) \end{array} \right.$ è continua. In particolare ogni spazio topologico compatto di Hausdorff $Y$ coincide con la compattificazione di Alexandroff di $Y \setminus \{y\} \quad \forall y \in Y$ 
	\end{itemize}
	
	\section*{Altri Lemmi}
	\begin{itemize}
		\item ({\bf Continuità e ricoprimenti fondamentali}) Sia $\cA$ un ricoprimento fondamentale di $X$. Un'applicazione $f:X \rar Y$ è continua $\sse \forall A \in \cA$ la restrizione $f\mid_A: A \rar Y$ è continua.
		\item ({\bf $[0,1]$ è tutto quanto}) L'intervallo $[0,1]$ per la topologia euclidea è connesso, connesso per archi, compatto, localmente connesso, localmente connesso per archi, localmente compatto.
		\item ({\bf Ricoprimenti localmente finiti}) I ricoprimenti aperti ed i ricoprimenti chiusi localmente finiti sono fondamentali.
	\end{itemize}	

	\section*{Sottospazi}
	\begin{itemize}
		\item ({\bf Passaggio della chiusura}) $Y \subseteq X$ sottospazio, $A \subseteq Y$. Allora la chiusura di $A$ in $Y$ è uguale all'intersezione di $Y$ con la chiusura di $A$ in $X$.
		\item ({\bf Passaggio di aperti-chiusi}) $Y \subseteq X$, $Z \subseteq Y$. Allora si hanno: \\
			\begin{itemize}
				\item $\se Y \apertoin X, \allora Z \apertoin Y \sse Z \apertoin X$
				\item $\se Y \chiusoin X, \allora Z \chiusoin Y \sse Z \chiusoin X$
				\item $\se Y \intornodi y, \allora Z \intornodi y \iin Y \sse Z \intornodi y \iin X$
			\end{itemize}
	\end{itemize}

	\section*{Topologie Comuni}
	\begin{itemize}
		\item ({\bf Topologia discreta}) $\tau = \cP (X)$ quindi ogni insieme è aperto. è indotta dalla distanza discreta: $d(x,y) = \left\{ \begin{array}{cr} 0 & \text{se } x = y \\ 1 & \text{se } x \neq y \\ \end{array} \right. $
		\item ({\bf Topologia indiscreta}) $\tau = \{\emptyset, X\}$, la meno fine tra tutte le topologie.
		\item ({\bf Topologia euclidea su $\bbR$}) Un sottoinsieme $U \subseteq \bbR$ è aperto se e solo se è unione di intervalli aperti.
		\item ({\bf Topologia della semicontinuità superiore di $\bbR$}) Gli aperti non vuoti sono tutti e soli i sottoinsiemi della forma $( - \infty , a)$, al variare di $a \in \bbR \cup \{+ \infty\}$
	\end{itemize}

	\section*{Metrizzabilità}
	\begin{itemize}
		\item ({\bf Proprietà di un metrico}) Sia $X$ spazio metrico. Allora $X$ è T2.
	\end{itemize}

	\section*{Che proprietà passano a cosa?}
	Vediamo alcune proprietà degli spazi \\
	Attenzione! Non vi fidate troppo delle cose in rosso perchè devo ancora verificare i risultati \\
	%<*TabellaProprieta>
	\begin{tabular}{lcccccc}
	{\bf Proprietà} & {\bf Sottospazi} & {\bf Prodotti} & {\bf Quozienti} & {\bf Funzioni $\cC^0$} & {\bf Implica} \\
	N1              & \rd\checkmark    & \rd\Numerabili &                 &                        &               \\ \hline
	N2              & \rd\checkmark    & \rd\Numerabili & \rd{Aperti}     & \rd{Aperte}            &               \\ \hline
	Sep             & \crossmark       & \Numerabili    &                 & \checkmark             &               \\ \hline
	T0              & \checkmark       & \Arbitrari     &                 &                        &               \\ \hline

	T1              & \checkmark       & \Arbitrari     &                 &                        &               \\ \hline
	T2              & \checkmark       & \Arbitrari     &                 &                        &               \\ \hline
	Reg             & \rd\checkmark    & \rd\Arbitrari  &                 &                        &               \\ \hline
	Norm            & \rd{Chiusi}      & \rd\crossmark  &                 &                        &               \\ \hline

	T3              & \rd\checkmark    & \rd\Arbitrari  &                 &                        &               \\ \hline
	T4              & \rd{Chiusi}      & \rd\crossmark  &                 &                        &               \\ \hline
	Cpt             & Chiusi           & \rd\Arbitrari  &                 & \checkmark             & (+T2) Chiuso  \\ \hline
	Lind            & Chiusi           & \rd\crossmark  &                 &                        &               \\ \hline

	Conn            & \crossmark       & \rd\Arbitrari  &                 & \checkmark             &               \\ \hline
	PathConn        & \crossmark       & \rd\Arbitrari  &                 & \checkmark             & Conn          \\ \hline
	LocConn         & \rd{Aperti}      &                &                 &                        &               \\ \hline
	LocPathConn     & \rd{Aperti}      &                &                 &                        &               \\ \hline

	Metr            & \checkmark       & \rd\Numerabili &                 &                        &               \\ \hline
	ParaCpt         & \rd{Chiusi}      & \rd\crossmark  &                 &                        &               \\ \hline
	\end{tabular} \vskip 1.5cm
	%</TabellaProprieta>

\end{document}
