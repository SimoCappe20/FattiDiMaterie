\documentclass[a4paper,NoNotes]{stdmdoc}

\newcommand{\B}{\mbox{B}}
\newcommand{\diam}{\mbox{ diam}}
\newcommand{\disc}{\mathfrak{Disc}}
\newcommand{\osc}{\mathfrak{Osc}}

\newcommand{\tc}{\mbox{ t.c. }}
\newcommand{\de}{\mbox{ d}}

\begin{document}
\title{Fatti generali di Analisi 1}

\Definizione{$F_\sigma$} Si dice $F_\sigma$ un sottoinsieme $I \subseteq \RR$ che sia unione numerabile di chiusi, ovvero se si può scrivere $$I = \bigcup_{n\in\NN} C_n$$

\Definizione{Insieme trascurabile} Si dice che un sottoinsieme $E \subseteq \RR$ è trascurabile (ovvero ha misura di Lebesgue nulla) se $\forall \varepsilon > 0 \quad \exists \{(a_n, b_n)\}_{n\in\NN} \tc E \subseteq \bigcup_{n\in\NN} (a_n, b_n)$

\Definizione{Funzione oscillazione} Di una funzione $f: \Omega \rightarrow \RR$ (dove $\Omega$ è uno spazio metrico) si definisce la funzione oscillazione $\Theta_{f}: \Omega \rightarrow \RR$ come: $$\Theta_{f}(x) := \lim_{r \rightarrow 0^{+}} \diam (f(\B_r(x)))$$ \\
Proprietà importanti: \begin{itemize}
	\item $\bar{x}$ è un punto di discontinuità $\Leftrightarrow$ $\Theta_f(\bar{x}) > 0$.
	\item $\Theta_f$ è una funzione semicontinua inferiormente.
	\item Definizione equivalente: $\Theta_f(x) = \left( \limsup_{y \rightarrow x} f(y) \right) - \left( \liminf_{y \rightarrow x} f(y) \right)$.
\end{itemize}

\Altro{Caratterizzazione della Riemann-integrabilità} Una funzione è Riemann-integrabile se e solo se l'insieme dei suoi punti di discontinuità è trascurabile.

\Altro{Teorema fondamentale del calcolo integrale, versione pro} $f: [a, b] \rightarrow \RR$ derivabile in $(a,b)$ e $f'$ Riemann-integrabile. Allora vale $f(b) - f(a) = \int_{a}^{b} f'(t) \de t$

\Altro{Teorema di Darboux} Le derivate mappano connessi in connessi. \\
Sia $f: [a,b] \rightarrow \RR$ ovunque derivabile, e si ponga $\alpha := f'(a), \beta := f'(b)$. Possiamo wlog supporre che $\alpha \le \beta$. Allora si ha, $\forall \alpha < \lambda < \beta \quad \exists \xi \in (a, b) \tc f'(\xi) = \lambda$.
\Dimostrazione Si consideri la funzione $g(x) := f(x) - \lambda x$. Questa funzione è continua (essendo $\lambda$ fissato e $f$ continua) e definita sul compatto $[a,b]$. Quindi ammette massimo e/o minimo. Siccome $g$ è anche derivabile si ha, nel punto di massimo $0 = g'(M) = f'(M) - \lambda \Rightarrow f'(M) = \lambda$.

\Altro{Punti di discontinuità di una funzione reale} Una funzione $f$ ha punti di discontinuità che sono un $F_\sigma$
\Dimostrazione Si consideri la funzione oscillazione di $f$: $\Theta_f(x)$. Fissata una "soglia di oscillazione" $\nu$ si ha che $\osc_f^{\ge \nu} := \{x \mid \Theta_f(x) \ge \nu\}$ è un chiuso (Si dimostri che se c'è un punto $y$ sul quale si accumula una successione $(y_n)$ di punti $\tc \Theta_f(y_n) \ge \nu$ allora si ha $\Theta_f(y) \ge \nu$). Ora, siccome i punti di discontinuità sono tutti e soli quelli con oscillazione maggiore di zero, si ha $\disc_f = \bigcup_{n\in\NN} \osc_f^{\ge \frac{1}{n}}$, ovvero unione numerabile di chiusi.

\Altro{Discontinuità di una funzione semicontinua inferiormente} I punti di discontinuità di una funzione semicontinua inferiormente sono di prima categoria (ovvero unione numerabile di chiusi a parte interna vuota).

\Altro{Prima Categoria - Misura di Lebesgue nulla} Non c'è nessuna implicazione tra queste due; ovvero esistono insiemi di prima categoria ma di misura positiva ed insiemi a misura nulla di seconda categoria.


\end{document}
