\documentclass[a4paper,NoNotes,GeneralMath]{stdmdoc}

\newcommand{\sgn}{\text{sgn }}

\begin{document}
	\title{Fatti di Analisi Numerica}
	
	\section*{Generalitate}
	\begin{itemize}
		\item ({\bf Numeri rappresentabili}) In un metodo numerico i numeri rappresentabili oltre allo zero sono tutti e soli quelli che si possono scrivere in una assegnata base $B$ con un numero finito $n$ di cifre e con un esponente, più precisamente come $$ \pm 0.c_1 c_2 \cdots c_n \times B^e $$ con il significato di $\pm B^e \cdot \sum_{i=1}^{n} c_i \cdot B^{-i}$, dove $c_1, \ldots, c_n$ sono le cifre, ovvero degli interi compresi tra $0$ e $B-1$.
		\item ({\bf Teorema di rappresentazione in base}) Per ogni numero reale $x \neq 0$ esistono unici un intero $p$ ed una successione $\{d_i\}_{i \ge 1}$ tali che $0 \le d_i \le B-1$, $d_1 \neq 0$ e $\forall k > 0 \quad \exists j \ge k \tc d_j \neq B-1$ per i quali quindi $$ x = \sgn(x) B^p \sum_{i=1}^{\infty} d_i B^{-i} $$
		\item ({\bf Errore inerente}) Assegnata una funzione $f: \Omega \subseteq \bbR^n \rar \bbR$ vorremmo calcolare $f(x)$ per un assegnato valore di $x \in \Omega$. Purtroppo dobbiamo accontentarci di calcolare $f(\tilde{x})$, dove $\tilde{x} \in \cF^n \cap \Omega$ è una $n$-upla di numeri macchina tali che $\tilde{x}_i = x_i (1 + \varepsilon_i)$, dove $\varepsilon_i$ sono gli errori di rappresentazione tali che $\mid \varepsilon_i \mid < u$. Ancora prima di iniziare abbiamo a che fare con l'errore relativo $$ \varepsilon_{in} = \frac{f(\tilde x) - f(x)}{f(x)} $$ definito se $f(x) \neq 0$
		\item ({\bf Errore algoritmico}) Supponiamo che $f: \bbR \rar \bbR$ sia razionale. Vogliamo calcolare il valore di $f(\tilde x)$ eseguendo un'opportuna sequenza di operazioni aritmetiche ciascuna delle quali introduce potenzialmente un errore locale. La funzione calcolata in verità sarà qualcosa di diverso da $f(\tilde x)$, in generale che indichiamo con $\varphi(\tilde x)$. Definiamo quindi l'errore algoritmico $$ \varepsilon_{alg} = \frac{\varphi(\tilde x) - f(\tilde x)}{f(\tilde x)} $$ generato dall'accumularsi degli errori locali relativi a ciascuna operazione aritmetica eseguita in floating point
		\item ({\bf Errore analitico}) Nel caso di una funzione non razionale $g(x): \Omega \rar \bbR$ dobbiamo selezionare una funzione razionale $f(x)$ che ben approssimi $g(x)$. Definiamo quindi l'errore analitico definito da $$ \varepsilon_{an} = \frac{f(x) - g(x)}{g(x)} $$
		\item ({\bf Errore totale}) $$ \varepsilon_{tot} = \frac{\varphi(\tilde x) - g(x)}{g(x)} $$ esprime di quanto il valore effettivamente calcolato $\varphi(\tilde x)$ si discosta dal valore $f(x)$ che avremmo voluto calcolare. $\varepsilon_{tot} .= \varepsilon_{in} + \varepsilon_{alg} + \varepsilon_{an}$
	\end{itemize}

	\section{Analisi degli errori}
	\begin{itemize}
		\item ({\bf Analisi dell'errore inerente}) Se $f: \bbR \rar \bbR$ è almeno $\cC^2 ([x, \tilde x])$. Considerando l'errore di rappresentazione $\delta_x = \frac{\tilde x - x}{x}$ si ricava $\varepsilon_{in} .= \delta_x \frac{x f'(x)}{f(x)}$. La quantità $ \delta_x \frac{x f'(x)}{f(x)}$ viene detta coefficiente di amplificazione. \\ Nel caso di $f: \bbR^n \rar \bbR$ vale una formula analoga per l'errore inerente. Infatti detto $x = (x_i), \delta_{x_i} = \frac{\tilde x_i - x_i}{x_i}$ risulta $\varepsilon_{in} .= \sum_{i=1}^{n} \delta_{x_i} C_i$, dove $C_i = \frac{x_i \dpar{f(x)}{x_i}}{f(x)}$ sono i coefficienti di amplificazione rispetto alla variabile $x_i$
		\item ({\bf Coefficienti di amplificazione delle operazioni aritmetiche}) \newline
			\begin{tabular}{l|cc}
			Operazione & $C_1$ & $C_2$ \\ 
			moltiplicazione & $1$ & $1$ \\
			divisione & $1$ & $-1$ \\
			addizione & $\frac{x_1}{x_1 + x_2}$ & $\frac{x_2}{x_1 + x_2}$ \\
			sottrazione & $\frac{x_1}{x_1 - x_2}$ & $-\frac{x_2}{x_1 - x_2}$ \\
			\end{tabular} \vskip 1.0cm
		\item ({\bf Analisi dell'errore algoritmico}) Ad ogni operazione compiuta bisogna propagare gli errori moltiplicandoli per i coefficienti di amplificazione (metodo in avanti)
		\item ({\bf Analisi all'indietro}) Consiste nel dire $\varphi(x_1, \ldots, x_n) = f(\hat x_1, \ldots, \hat x_n)$
	\end{itemize}

\end{document}
