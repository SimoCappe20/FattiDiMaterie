\documentclass[a4paper,NoNotes,GeneralMath]{stdmdoc}

\begin{document}
	\title{Fatti di Analisi 2}
	\author{}

	\section*{Convergenze Varie}
	\begin{itemize}
		\item ({\bf Puntuale}) Una successione di funzioni $f_n(x)$ converge puntualmente a $f(x)$ se $\forall x \quad \forall \varepsilon > 0 \quad \exists n_0 \tc \forall n \ge n_0 \qquad \abs{f_n(x) - f(x)} \le \varepsilon$
		\item ({\bf Uniforme}) Una successione di funzioni $f_n(x)$ converge uniformemente a $f(x)$ se $\forall \varepsilon > 0 \quad \exists n_0 \tc \forall n \ge n_0 \quad \forall x \qquad \abs{f_n(x) - f(x)} \le \varepsilon$
		\item ({\bf Assoluta}) Una serie di funzioni $\Sigma_{n=0}^{+\infty} f_n(x)$ converge assolutamente se le serie $\Sigma_{n=0}^{+\infty} \abs{f_n(x)}$ converge puntualmente
		\item ({\bf Totale / Normale}) Una serie di funzioni $\Sigma_{n=0}^{+\infty} f_n(x)$ converge totalmente (al suo limite) in $A$ se vale che $\Sigma_{n=0}^{+\infty} \sup_{x\in A} \abs{f_n(x)} < +\infty$
		\item Assoluta $\implies$ Puntuale
		\item Uniforme $\implies$ Puntuale
		\item Totale $\implies$ Uniforme, Assoluta
	\end{itemize}

	\section*{Passaggio al Limite}
	Nel seguito si usa $f_n(x)$ per indicare una generica successione di funzioni, $f(x)$ il suo limite (dove esiste)
	\begin{itemize}
		\item ({\bf Continuità del Limite}) Se le $f_n(x)$ definitivamente sono continue, e la convergenza è uniforme, allora $f(x)$ è continua.
		\item ({\bf Derivabilità}) 
		\item ({\bf Integrabilità}) 
	\end{itemize}

	\section*{Problemi di Cauchy}
	Nel seguito parliamo di un problema del seguente tipo:
	\system{y' = f(x,y)}{y(x_0) = y_0}\par
	\begin{itemize}
		\item ({\bf Esistenza ed Unicità Locali}) 
		\item ({\bf Teorema di Peano, Esistenza Locale}) 
	\end{itemize}
\end{document}
