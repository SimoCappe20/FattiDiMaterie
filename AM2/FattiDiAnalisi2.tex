\documentclass[a4paper,NoNotes,GeneralMath]{stdmdoc}

\newcommand{\de}{\mbox{  d}}

\begin{document}
	\title{Fatti di Analisi 2}
	\author{}

	\section*{Convergenze Varie}
	\begin{itemize}
		\item ({\bf Puntuale}) Una successione di funzioni $f_n(x)$ converge puntualmente a $f(x)$ se $\forall x \quad \forall \varepsilon > 0 \quad \exists n_0 \tc \forall n \ge n_0 \qquad \abs{f_n(x) - f(x)} \le \varepsilon$
		\item ({\bf Uniforme}) Una successione di funzioni $f_n(x)$ converge uniformemente a $f(x)$ se $\forall \varepsilon > 0 \quad \exists n_0 \tc \forall n \ge n_0 \quad \forall x \qquad \abs{f_n(x) - f(x)} \le \varepsilon$
		\item ({\bf Assoluta}) Una serie di funzioni $\Sigma_{n=0}^{+\infty} f_n(x)$ converge assolutamente se le serie $\Sigma_{n=0}^{+\infty} \abs{f_n(x)}$ converge puntualmente
		\item ({\bf Totale / Normale}) Una serie di funzioni $\Sigma_{n=0}^{+\infty} f_n(x)$ converge totalmente (al suo limite) in $A$ se vale che $\Sigma_{n=0}^{+\infty} \sup_{x\in A} \abs{f_n(x)} < +\infty$
		\item Assoluta $\implies$ Puntuale
		\item Uniforme $\implies$ Puntuale
		\item Totale $\implies$ Uniforme, Assoluta
	\end{itemize}

	\section*{Passaggio al Limite}
	Nel seguito si usa $f_n(x)$ per indicare una generica successione di funzioni, $f(x)$ il suo limite (dove esiste)
	\begin{itemize}
		\item ({\bf Continuità del Limite}) Se le $f_n(x)$ definitivamente sono continue, e la convergenza è uniforme, allora $f(x)$ è continua.
		\item ({\bf Derivabilità del Limite}) Se le $f_n(x)$ convergono in un punto $\bar{x}$ ad $f(\bar{x})$ e le derivate $f_n'(x)$ convergono uniformemente ad una funzione $g(x)$ allora si ha che le $f_n(x)$ convergono uniformemente ad una funzione derivabile $f(x)$ tale che $f'(x) = g(x)$
		\item ({\bf Integrabilità del Limite}) Se le $f_n(x)$ convergono uniformemente alla $f(x)$ limite allora si ha $\lim_{n \rightarrow \infty} \int f_n(t) \de t = \int f(t) dt$
	\end{itemize}

	\section*{Problemi di Cauchy}
	Nel seguito parliamo di un problema del seguente tipo:
	\system{y' = f(x,y)}{y(x_0) = y_0}\par
	Con $f: U \rightarrow \RR^n$ è una funzione continua definita su un aperto.
	Indicheremo una generica soluzione con $\varphi: I \rightarrow \RR^n$ di classe $\mathcal{C}^1$ con $x_0 \in I$, tale che $\forall x \in I \quad (x, \varphi(x)) \in U$ e $\varphi'(x) = f(x, \varphi(x))$ e che $\varphi(x_0) = y_0$
	\begin{itemize}
		\item ({\bf Cauchy-Lipschitz, Esistenza ed Unicità Locali}) Se $f$ è continua e localmente lipschitziana in $y$ uniformemente rispetto a $x$, allora $\exists! \varphi$ soluzione {\bf locale} di classe $\mathcal{C}^1$
		\item ({\bf Teorema di Peano, Esistenza Locale}) Per garantire l'esistenza locale (ma non l'unicità!) basta che $f$ sia continua
		\item Si assuma che l'equazione $y' = f(x,y)$ abbia esistenza ed unicità locale in ogni punto di $U$. Allora si ha \\ {\bf Unicità Globale} (ovvero se due soluzioni coincidono in un punto allora coincidono in tutto l'intervallo); \\ {\bf Dominio Aperto delle soluzioni massimali} (una soluzione massimale ha come dominio un intervallo aperto); \\ {\bf Fuga dai compatti delle soluzioni massimali} (ovvero una soluzione massimale esce definitivamente da ogni sottoinsieme compatto di $U$)
		\item ({\bf Esistenza e Unicità Globale}) Supponendo che la funzione $f$ sia continua e localmente lipschitziana rispetto a $y$ (ovvero ipotesi di Cauchy-Lipschitz) e se si ha che per ogni intervallo compatto $K$: $\exists A_K, B_K > 0 \qquad \Norm{f(x,y)} \ge A_K \Norm{y} + B_K \quad \forall (x,y) \in K \times \RR^n$ allora per ogni $(x_0, y_0) \in I\times \RR^n$ il problema ha una ed una sola soluzione definita su {\it tutto} $I$ (supponiamo abbia soluzione limitata, allora deve fuggire dai compatti dove la $f$ è definita, ma se prendiamo il compatto delimitato dal bound della lipschitzianità, la funzione non può fuggirne localmente)
	\end{itemize}

	\section*{Soluzione di Equazioni Differenziali Comuni}
	\begin{itemize}
		\item ({\bf Lineari del prim'ordine}) Data l'equazione $y' = a(x)y(x) + b(x)$ (detta $A(x) = \int_{x_0}^{x} a(t) \de t$ una primitiva di $a(x)$) si ottiene, moltiplicando entrambi i membri per $e^{A(x)}$, la soluzione generale $y(x) = e^{A(x)} \int_{x_0}^{x} e^{-A(t)}b(t) \de t + c$
		\item ({\bf A variabili separabili}) Data l'equazione $y' = g(x)f(y)$ si ha $\int \frac{\de y}{f(y)} = \int g(x) \de x$ e calcolando le primitive si risolve il problema
	\end{itemize}
\end{document}
